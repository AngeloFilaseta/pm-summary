\section{Definizione di Project Management}
\subsection{Definizione di Project Management}
\begin{itemize}
	\item \textbf{Project Management (PMBOK)}: corrisponde all'applicazione di conoscenze, capacità, strumenti, tecniche alle attività di progetto per soddisfare i requisiti.
	\item \textbf{Project Management (Wysocki)}: corrisponde ad un insieme di strumenti, templates e processi studiati per rispondere alle seguenti domande:
	\begin{itemize}
		\item Quale situazione aziendale affronta il progetto?
		\item Cosa hai bisogno di fare?
		\item Cosa farai?
		\item Come lo farai?
		\item Come saprai che lo hai fatto?
		\item Quanto bene lo hai fatto?
	\end{itemize}
	\item \textbf{Project Management (Wysocki \#2)}: un approccio organizzato basato sul "buon senso" (organized common-sense approach) che deve prevedere un adeguato coinvolgimento del cliente al fine di soddisfare i requisiti definiti fornendo l'incremento di business value atteso.
	\begin{info}
		Secondo questa definizione Wysocki vuole dire che la responsabilità della definizione di business value è del cliente, attraverso la specifica dei requisiti. In realtà si tratta di una condivisione di responsabilità, da una parte il cliente può spiegarsi male o possiamo essere noi a capire male cosa vuole intendere. Il Project Manager resta il responsabile del soddisfacimento dei requisiti.
	\end{info}
\end{itemize}
\subsubsection{Domande sui progetti}
\begin{itemize}
	\item \textbf{Quale situazione aziendale affronta il progetto?}: Può consistere in un problema da risolvere o in un'opportunità non ancora sfruttata.
	\begin{info}[Esempio]
		\textbf{Problema}: un ordine su dieci arriva in ritardo $\xrightarrow{}$ si crea un progetto in modo tale che il numero di ordini soddisfatti passi dal 10\% all' 1\%.\newline
		\textbf{Opportunità}: abbiamo venduto tutto il vendibile, saturando il mercato coi nostri prodotti $\xrightarrow{}$ con un'analisi del mercato possiamo capire come valorizzare ulteriormente il prodotto.
	\end{info}
	In molte situazione non è facile definire con precisione la soluzione. Questo è un alto fattore di rischio.
	\item \textbf{Cosa hai bisogno di fare?}: Se il deliverable è definito chiaramente probabilmente è semplice definire cosa è necessario fare. Questo non significa che non possono esserci problemi sulle modalità di esecuzione (ad esempio la scelta delle tecnologie). Nei casi in cui il delivrable non è chiaro conviene utilizzare un approccio che prevede iterazioni, in modo che la convergenza verso i requisiti sia più lenta ma meno rischiosa.
	\item \textbf{Cosa si farà?}: A questo punto il "cosa fare" è conseguenza diretta dei passi precedenti. Può essere utile avere un \textit{Project Overviw Statement (POS)} o un \textit{Project Charter}. Vedremo meglio di cosa si tratta in futuro.
	\item \textbf{Come si farà?}: Questa è la parte di planning, si definisce l'approccio di progetto e si redige il piano delle attività necessarie.
	\item \textbf{Come capire se è stato fatto?}: Il deliverable deve fornire business value, cioè soddifsare dei criteri di successo. Alcuni di questi possono essere per esempio:
	\begin{itemize}
		\item \textbf{Increased Revenue}: aumento del fatturato;
		\item \textbf{Avoided Cost}: riduzione dei costi;
		\item \textbf{Improved Service}: miglioramento del servizio.
	\end{itemize}
	Conviene sempre esprimere quantitativamente le componenti del business value. La misurabilità è come al solito molto importante. Un errore comune quando si definiscono le componenti di business value è specificare indici che non dipendono dalla soluzione adottata (l'implementazione di un e-commerce non necessariamente farà aumentare il fatturato).
	\item \textbf{Quanto bene lo hai fatto?}:
\end{itemize}
\subsection{Cosa sono i requisiti?}
Seguono alcune efinizioni di \textbf{Requisito}:
\begin{itemize}
	\item \textbf{Requisito (IEEE Std 610.12)}: 
	\begin{enumerate}
		\item Una condizione o una capacità di cui necessita un utente (stakeholder) per risolvere un problema o arrivare ad un obiettivo \textit{(scope)};
		\item Una condizione o una capacità che deve essere posseduta o soddisfatta da un sistema o da un suo componente al fine di soddisfare un contratto, uno standard, una specifica o qualsiasi formalità espressa in un documento \textit{(soluzione)}.
		\item Una rappresentazione in documento di una condizione o capacità come specificato nei punti precedenti.
	\end{enumerate}
	\begin{warn}
		Esistono progetti in cui si può avere la massima chiarezza sullo scope ma nessun tipo di idea su come arrivare alla soluzione. Ci sono anche casi in cui nemmeno lo scope è chiaro (ad esempio ricerca e sviluppo). Al massimo quello che si ha in questo caso è l'ambito. Ovviamente lo Scope è più importante della Solution.
	\end{warn}
	\item \textbf{Requisito (Wysocki)}: Uno stato finale desiderato, la cui integrazione con successo nella soluzione fornisce all'organizzazione un aumento specifico e misurabile di business value.
	\begin{info}
		Il numero dei requisiti e su una scala completamente diversa (6-8). I requisiti non sono propriamente meno, ma sono raggruppati in gerarchie in cui il primo livello è più generale ed importante .Partendo dai requisiti si inizia a lavorare, e mano a mano che si scende in un livello di dettaglio maggiore. La ricerca della soluzione è incentrata completamente sul business value. Il focus permette di utilizzare nel miglior modo possibile le risorse.
	\end{info}
\end{itemize}
\subsection{Project Management Life Cycles}
Il \textbf{Project Management Life Cycle Model} è definibile come il ciclo di vita della gestione di un progetto ed è modellato utilizzando una sequenza di processi raggruppabili in 5 gruppi:
\begin{itemize}
	\item \textbf{Initiating/Scoping}: definizione dell'ambito del progetto;
	\item \textbf{Planning}: pianificazione;
	\item \textbf{Launching/Execution}: avvio ed esecuzione del progetto;
	\item \textbf{Monitoring \& Controlling}: monitoraggio e controllo;
	\item \textbf{Initiating/Scoping}: chiusura del progetto;
\end{itemize}
Di questi cinque gruppi, ciascuno di essi devono comparire almeno una volta nella sequenza per il raggiungimento degli obiettivi. Alcuni possono essere ripetuti più volte.
\begin{info}
	I due processi più importanti ed onerosi sono \textit{Initianting/Scoping} e \textit{Planning}, che saranno anche quelli su cui ci concentreremo di più. Il processo di \textit{Monitoring \& Controlling} è invece sicuramente il più pervasivo e sempre presente, anche se sarà conseguenza diretta dei primi due già menzionati. Proprio per questo è molto importante dare loro importanza.
\end{info}
\centeredImage{document/img/pmlc.PNG}{Project Management Life Cycles - Dove può cadere un progetto?}{0.5}
\subsection{Traditional Project Management (TPM)}
Si tratta del Project Management tradizionale, ovvero l'approccio che segue il modello a cascata tradizionale. Avendo le idee molto chiare sia su scope che su soluzione i rischi sono sempre bassi. Di solito se ci si trova qui è perché si conosce bene il problema e il modo in cui va affrontato. 
\subsection{Agile Project Management (APM)}
Nel caso in cui il goal è chiaro ma non si riesce ad inquadrare bene la soluzione può essere utile utilizzare un approccio Agile. C'è ancora dibattitto sul fatto che alcune metodologie Agile funzionino meglio avendo chiara la soluzione e non lo scope. Iterando è possibile creare la soluzione a piccoli step, in modo da poter confrontare la soluzione insieme al cliente. In questo caso il problema è la convergenza verso la fine del progetto.
\subsection{Extreme Project Management (xPM)}
Da non confondere con l'approccio denominato \textit{Extreme Programming}. Extreme è una keyword usata spesso per descrivere la situazione in cui non si hanno le idee chiare ne sul goal ne sulla soluzione. Un classico esempio è il campo della ricerca e dello sviluppo. Un altro esempio può essere il reengineering di un processo aziendale che va male, ma che l'azienda non sa come riprogettare. Il focus qui è trovare soluzione innovative.
\subsection{Emertxe Project Management (MPx)}
Questa soluzione viene utilizzata quando si hanno le idee chiare sulla soluzione da adottare ma non sullo scope. Per esempio quando è obbligatorio e vincolante l'utilizzo di una certa tecnologia ma non quale sia il vero fine. Un esempio è Facebook, in cui l'architettura e il funzionamento della piattaforma sono sempre stati chiari, ma non il "cosa me ne faccio?".
\begin{info}
	Per qualche motivo si pronuncia "E-murt-see".
\end{info}
\subsubsection{Approcci di PMLC}
\centeredImage{document/img/pmlcapproach.PNG}{Project Management Life Cycles - Approcci}{0.5}
Potremmo trovarci in un punto qualsiasi di questo quadrante. L'approccio da utilizzare varia in base a dove ci troviamo.
\paragraph{Linear Project Management Life Cycle Model}
Una volta definito lo scope, dato che ho un livello di certezza molto altro sullo scope non tornerò mai indietro. Si concretizzerà poi la fase di pianificazione, scegliendo i dettagli tecnologici. Alla fine di questa fase ho già definito la solution. Non sono previste modifiche rilevanti allo scope. Si utilizzano templates consolidati.
\centeredImage{document/img/linear.PNG}{Project Management Life Cycles - Approccio lineare}{0.9}
\paragraph{Incremental Project Management Life Cycle Model}
Molto simile all'approccio lineare, è tutto già pianificato, ma si procede a step incrementali. Il vantaggio è poter apportare piccole modifiche rispetto a release predenti. Serve leggermente più flessibilità rispetto all'aproccio lineare. Anche in questo caso lo scope ha una basssa probabilità su subire modifiche.
\centeredImage{document/img/incremental.PNG}{Project Management Life Cycles - Approccio incrementale}{0.9}
\paragraph{Iterative Project Management Life Cycle Model}
Siamo già nel mondo Agile. In questo caso lo scope non si tocca e si definisce una sola volta, ma non si crea un piano unico iniziale. La pianificazione si esegue per esempio all'inizio di ogni sprint. Il problema è la definizione di un contratto col cliente che non è facilmente redigibile essendoci molta incertezza sulla soluzione. Si apprende meglio il dominio mano a mano che si crea l'implementazione.
\centeredImage{document/img/iterative.PNG}{Project Management Life Cycles - Approccio iterativo}{0.9}
\paragraph{Adaptive Project Management Life Cycle Model}
L'obiettivo è noto, ma non la soluzione. La soluzione è fortemente influenzata dai cambiamenti attesi. Di solito cadono in questo approccio la progettazione di nuovi prodotti o il miglioramento dei processi.
\paragraph{Extreme Project Management Life Cycle Model}
In questo campo l'iterazione avviene anche ripensando allo scope.
\centeredImage{document/img/extreme.PNG}{Project Management Life Cycles - Approccio extreme}{0.9}
\subsubsection{Modello PMLC proposto da PMBOK}
Il modello proposto da PMBOK è molto semplice, e comprende una fase di inizio progetto in cui si inizializzano i processi dpo aver iniziato il progetto. Si susseguono il planning dei processi e la loro esecuzione in maniera continua finché non finiscono e si converge quindi ad un risultato, ossi la fine dle progetto. Questo ciclo di vita può essere valido per un progetto di qualsiasi tipo.
\centeredImage{document/img/pmlcpmbok.PNG}{Project Management Life Cycles - Secondo PMBOK}{0.9}
\subsection{Come scegliere il best fit PMLC Model}
Non è possibile scegliere il modello PMLC migliore senza prima aver eseguito la fase di scoping. Una volta aver finito la pianificazione possiamo capire come il progetto va gestito e quali sono le effettive specifiche.
\centeredImage{document/img/choosepmlcmodel.PNG}{Project Management Life Cycles - Scelta del modello}{0.9}
\subsubsection{Ruolo del Project Manager}
Il Project Manager è la figura professionale che si occupa di guidare il team per raggiungere gli obiettivi. In molte aziende il PM è anche l'architetto o chi detiene il prodotto. Il problema in questo caso è che ci sono dei rischi dovuti al conflitto di interesse. La cosa migliore è tenere la figura del Project Manager e quella di chi detiene la responsabilità tecnica del prodotto separate.\newline
In un team Scrum il Project Manager potrebbe essere per esempio il Product Owner. Ogni azienda ha la sua soluzione specifica, ed è possibile un ampio margine di personalizzazione. In alcune strutture organizzative il Project manager dipende da un ufficio progetti. I project manager hanno la responsabilità di soddisfare i bisogni del team e degli individui relativi alle attività del progetto.\newline
I progetti sono essenziali per la crescita e la sopravvivenza delle aziende, perché creano valore. Secondo PMBOK un Project Manager deve avere le seguenti caratteristiche:
\begin{itemize}
	\item \textbf{Knowledge}: il Project Manager deve avere conoscenze in project management.
	\item \textbf{Performance}: il Project Manager è capace di applicare la conoscenza del Project Management.
	\item \textbf{Personal}: il Project Manager deve sapere come comportarsi in un progetto. Deve possedere delle soft-skill per guidare il team e gestire bene i rapporti interpersonali.
\end{itemize}
\newpage