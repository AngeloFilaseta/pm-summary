\section{Launching/Execution Process Group}
\subsection{Tools, templates e processi per avviare ed eseguire un progetto}
Segue una lista di tool templates e processi utili per avviare un progetto:
\begin{itemize}
	\item Recruiting/staffing del Project Team (processo);
	\item Project Definition Statement (template);
	\item Regole operative per il team (Team Operating Rules):
	\begin{itemize}
		\item Problem Solving (tool, template e processo);
		\item Decision Making (tool);
		\item Conflict Resolution (tool);
		\item Consensus Building (tool);
		\item Brainstorming (processo);
		\item Team Meetings (tool).
	\end{itemize}
	\item Gestione del processo di modifica dello Scope (template e processo);
	\item Communications Management Planning (processo);
	\item Work Packages (tool e template);
	\item Resource assignment (processo);
	\item Raffinamento della schedula del progetto (template e processo);
\end{itemize}

\subsection{Recruiting the project team}

La domanda che viene fatta sempre è "Quanto ci si mette a staffare?" ossia "Quanto ci vuole a creare il team?".

Un project team è un gruppo di professionisti impegnati a raggiungere degli obiettivi comuni, che dovrebbero essere affiatati tra loro per lavorare bene assieme, relazionandosi direttamente e apertamente per fare ciò che è stato
stabilito.

Le figure che sono indispensabili sono:
\begin{itemize}
	\item \textbf{Co-Project manager - lato developer}
	\item \textbf{Co-Project manager - lato cliente}: ci dovrebbe essere. A volte non c'è ma sarebbe utile perché in due è più facile monitorare e gestire il progetto.
	\begin{info}
		Di solito vengono scelti prima della fase di Scoping. A volte (raramente) non vengono utilizzati gli stessi project manager che hanno seguito la fase di scoping. Ovviamente questa non è una situazione ideale).
	\end{info}
	\item \textbf{Core team}: ha fatto le stime e ha deciso le attività, sa cosa deve essere implementato e quando.
	\begin{info}
		Per questo motivo il core team viene scelto prima della fase di Planning. Non sarebbe male riuscire a confermarlo in esecuzione. Probabilmente qualche componente è allocato anche in altri progetti e non riuscirà a partecipare al 100\%. È comunque ottimale riuscire a massimizzare la presenza dello stesso core team che ha gestito la situazione durante la fase di planning.
	\end{info}
	\item \textbf{Developer team}: Figure che vengono a conoscenza del progetto solo una volta che è necessario partire. Importante cercare di rendere chiare le idee fin da subito.
	\item \textbf{Client team}: Variegato in base alla difficoltà del progetto. Sono le figure che fanno review e che ci aiutano a prendere decisioni.
	\item \textbf{Contracted team}: Un team (scelto sul mercato) che aiuta a gestire il progetto. Si tratta di personale che viene acquisito tramite un contratto e si occupa del progetto come "esterno", magari perché c'è poco tempo e serve più personale o molti altri motivi.
	\begin{info}
		Queste figure vengono scelte in fase di Launching.
	\end{info}
\end{itemize}

\subsubsection{Caratteristiche del core team e del team di sviluppo}
I membri del team, in particolare del core team e del team che svilupperà la
soluzione, dovrebbero avere le seguenti caratteristiche:
\begin{itemize}
	\item \textbf{Impegno}: persone motivate e che saranno focalizzate sul progetto. Si tratta del minimo sindacale. Persone che hanno poco impegno non renderanno bene;
	\item \textbf{Condivisione delle responsabilità}: Capire cosa è importante e capire quali sono le responsabilità del team, il quale dovrà farsene carico;
	\item \textbf{Flessibilità}: adattarsi a cambiare mansione. C'è chi ama fare sempre le stesse cose, ma avere flessibilità significa riuscire a massimizzare il modo in cui le attività vengono gestite dal personale;
	\item \textbf{Task-oriented}: I membri del team devono capire lo scopo di ogni task in modo da mantenere l'integrità concettuale del progetto;
	\item \textbf{Abilità a lavorare entro i tempi e i vincoli previsti}: Il piano deve essere seguito, serve disciplina soprattutto nelle fasi iniziali del progetto, dato che è facile rallentare data la mancanza di pressione. La gestione del tempo è fondamentale e deve essere gestita bene dall'azienda;
	\item \textbf{Propensione a concedere fiducia e supporto reciproco}: Ogni sviluppatore ha la sua opinione, ma serve anche dare fiducia a delegare e a supportarsi in caso di necessità;
	\item \textbf{Team-oriented}: collaborazione, abilità a comunicare in modo corretto;
	\item \textbf{Open-minded}: saper capire le ragioni degli altri e sintetizzare il miglior risultato;
	\item \textbf{Abilità a lavorare in ambienti strutturati rispettando le gerarchie}: la decisione va comunque presa da chi ha più autorità, anche nei team più orizzontali. Si tratta di un modo per convergere anche se magari spesso non si sceglie nel modo migliore possibile;
	\item \textbf{Abilità a usare strumenti per il project management}: conoscere i termini, le buone pratiche, acquisirle, adattarle ai nuovi progetti;
\end{itemize}

\subsubsection{Altri membri del team}
A volte i membri del team che seguono non ci sono, ma è necessario avere certe aspettative nel caso vengano coinvolti:
\paragraph{Client team}
\begin{itemize}
	\item \textbf{Devono comprendere i processi della loro business unit}: se abbiamo un dubbio, il client team è il punto di riferimento principale. Devono conoscere i processi della loro business unit. Necessario minimizzare i casi in cui il client team ha bisogno di effettuare richieste all'azienda per avere risposte a domande che gli vengono poste;
	\item \textbf{Devono essere in grado di prendere decisioni e impegni per conto della loro business unit}: devono avere l'autorità di poter prendere decisioni per conto dell'azienda.
\end{itemize}
\begin{info}
	Abbiamo questo tipo di aspettative perché vogliamo evitare di perdere tempo. Se il client team non rispetta questi canoni possiamo perdere giorni e non riusciremmo a rispettare la schedula. Questo renderebbe il client team completamente inutile. Il suo scopo principale infatti è supportarci quando abbiamo dubbi e abbiamo bisogno di chiarimenti.
\end{info}

\paragraph{Contract team}
\begin{itemize}
	\item \textbf{Potrebbero non essere sempre disponibili quando sono necessari}: nel contratto stabiliamo come possiamo utilizzare i membri del team. Questo tipo di figure potrebbero essere impegnati in altre mansioni (magari per altri progetti) e potrebbero non essere sempre disponibili quando ne abbiamo bisogno;
	\item \textbf{Dovrebbero conoscere come i task assegnati sono relazionati con gli altri task e l’intero progetto}: Spesso gli esterni non hanno molto interesse a capire bene come funziona il progetto e come i task sono legati tra loro perché rimangono comunque figure esterne che prima o poi spariranno. Questo è un problema notevole per l'azienda per cui è necessario cercare di evitare che vengano create tante piccole incongruenze che poi portano ad un refactor necessario, che non sempre tra l'altro è possibile;
	\item \textbf{L’impegno può essere un problema}: il contract team fattura ad ore, quindi spesso queste figure non hanno fretta di consegnare qualcosa di funzionante;
	\item \textbf{La qualità del lavoro svolto può essere scarsa}: per le ragioni già citate, ma anche perché figure esterne non possono seguire le policy dell'azienda come l'azienda stessa;
	\item \textbf{Richiedono una supervisione più stretta rispetto al core team}: per gli ovvi motivi già espressi.
\end{itemize}

\subsubsection{Bilanciamento di un Team}
Trovare il giusto bilanciamento nella composizione di un team è un “fattore critico di successo”.
In letteratura sono state proposte numerose tecniche per misurare e stabilire il giusto bilanciamento. Il modo in cui il team va bilanciato dipende fortemente dal tipo di progetto. Un esempio che David Kolb ha proposto: il seguente è un metodo di classificazione in termine di "learning style":
\begin{itemize}
	\item \textbf{Assimilating}: Gli assimilatori sono molto bravi a raccogliere e rappresentare i dati in modo brillante. Preferiscono fornire dei modelli che descrivono la realtà da una prospettiva “più larga”. Si focalizzano su idee e concetti piuttosto che sulle persone e gli aspetti pratici. La parte negativa è che NON sono individui orientati ai risultati. In pratica si tratta di figure più orientate alla fase di analisi, alla progettazione etc. ma non sono molto utili quando è il momento di scrivere molto codice.
	\item \textbf{Diverging}: Sono individui che hanno una naturale propensione a individuare alternative e vedono le diverse situazioni da molteplici prospettive. Sono le figure che quando fanno le stime riescono a vedere molteplici opzioni alternative. Tendono più ad osservare che ad agire. Preferiscono il brainstorming all’azione. Possono fornire al team di sviluppo però dei punti di vista originali, non convenzionali (outside-of-the-box), che possono permettere al team l’individuazione di approcci alternativi.
	\item \textbf{Accommodating}: Sono individui orientati ai risultati, che preferiscono l’azione. Si adattano facilmente in base alle circostanze. Tendono a relazionarsi con gli altri per decidere le loro azioni, piuttosto che impiegare analisi tecniche. La loro azione è molto positiva per il team. In pratica è una figura che "fa ciò che gli viene detto".
	\item \textbf{Converging}: Sono individui orientati ai risultati. Preferiscono individuare le soluzioni che la loro effettiva implementazione. Non sono particolarmente orientati alla collaborazione con gli altri membri e preferiscono lavorare ad attività e problemi tecnici. Sono molto abili a individuare la migliore opzione tra diverse alternative. Guidano il team alla scelta del giusto approccio. Si chiamano così proprio perché "convergono" alla soluzione del problema.
\end{itemize}
Ovviamente la realtà è molto più variegata di così, ma una classificazione sommaria può essere utile.

\subsubsection{Cosa rende efficace un Team}
Le principali caratteristiche, che permettono a un team di essere efficace,
sono le seguenti:
\begin{itemize}
	\item \textbf{Qualità}: imparare ad aumentare la qualità rispettando standard e policy;
	\item \textbf{Flessibilità}: di skill, di abitudine etc.;
	\item \textbf{Coordinamento e cooperazione}: non tanto importanti i singoli che cercano di fare il meglio quanto la collaborazione con gli altri membri del team;
	\item \textbf{Soddisfazione dei membri}: fattori motivanti per rendere più felici e produttive le persone;
	\item \textbf{Crescita professionale}: variegazione dei tipi di lavori che vengono attributi alle persone per farle crescere;
	\item \textbf{Possesso delle abilità e competenze necessarie}: se ci sono lacune non è un problema solo del team ma anche dell'azienda. Sempre utile fornire corsi, supervisione e altri tipi di supporto.
	\item \textbf{Produttività}: più e alta e meglio è. Se il team ha le caratteristiche sopra citate allora probabilmente sarà più produttivo.
\end{itemize}

\subsubsection{Miglioramento continuo in un Team}
Nei processi di miglioramento continuo di un’azienda dovrebbe essere posta particolare attenzione ai seguenti aspetti:
\begin{itemize}
	\item \textbf{Metodi e procedure di lavoro}: nella retrospettiva bisogna sempre discutere se i metodi e le procedure hanno funzionato correttamente, parlare dei problemi che ci sono stati, trovare metodi per migliorare;
	\item \textbf{Conoscenza delle tecnologie più appropriate per l’implementazione delle soluzioni}: se non si ha a disposizione abbastanza supporto può essere utile il training o l'affiancamento di un esperto;
	\item \textbf{L’uso di servizi e prodotti di qualità}: un ambiente di lavoro non deve disporre di troppe "noie tecniche". Necessaria un'infrastruttura che riesca a gestire le operazioni più semplici e ripetitive nel modo più rapido possibile;
	\item \textbf{Decision making}: il come prendere decisioni è difficile. Migliorare questo processo è una fase molto delicata e va affrontata in modo molto approfondito;
	\item \textbf{Supervisione}: se non ce n'è abbastanza va aumentata. Non si può rinunciare alla supervisione, meglio dare più spazio e promuoverla;
	\item \textbf{Supporto allo staff}: avere una segreteria che gestisce fasi più pratiche come l'alloggio in hotel o la prenotazione di biglietti per i viaggi permette di usare meglio il proprio tempo;
	\item \textbf{Attrarre e fidelizzare}: niente di peggio di gente che va via durante lo sviluppo di un progetto. Trattenere le persone e motivarle è cruciale;
	\item \textbf{Tasso di uscita (per ridurlo il più possibile)}: tassi di uscita alti vanno abbassati;
	\item \textbf{Flessibilità nella composizione del personale}: il personale deve offrire più flessibilità possibile;
\end{itemize}

\subsubsection{Responsabilità dei membri del Team}
Tutti i membri del team hanno delle precise responsabilità, in particolare
nei seguenti ambiti:
\begin{itemize}
	\item \textbf{Comunicazione}: per poter avere buone comunicazione, tutti i membri del team devono comunicare quando necessario e nel modo corretto (ad esempio meglio comunicare ritardi in ufficio o scegliere attentamente cosa dire al cliente perché poi è difficile tornare indietro);
	\item \textbf{Attitudine all’ascolto}: è difficile ascoltare, spesso magari viene naturale non ascoltare perché si crede di avere le idee già chiarissime;
	\item \textbf{Condivisione degli obiettivi}: bisogna essere concordi e viaggiare verso la stessa direzione. È nostra responsabilità fare in modo che questo avvenga;
	\item \textbf{Atteggiamento positivo}: essere positivi è contagioso e permette anche agli altri di lavorare meglio;
	\item \textbf{Creatività}: non è solo una attitudine personale, a volte essere creativi può portare ad ottime soluzioni. Questo non significa non seguire standard e policy aziendali;
	\item \textbf{Rispetto degli altri}: quando emerge un problema è nostra responsabilità cercare di mantenere tutto sotto controllo. La collaborazione risolve i problemi, la rabbia no;
	\item \textbf{Crescita e apprendimento}: ad esempio seguire un corso con la dovuta attenzione, studiare il materiale fornito dalla supervisione etc.
\end{itemize}

\subsection{Conducting the project kick-off meeting}
Il Project Kick-Off Meeting provvede ad annunciare che il progetto pianificato è stato approvato per la fase esecutiva. La riunione per il kick-off solitamente si risolve in un solo incontro e serve essenzialmente per far incontrare i “partecipanti”. Al progetto potrebbero partecipare attivamente anche solo 10 persone, ma alla riunione di kick-off sono presenti molte più figure, ad esempio gli stakeholder. Ricordiamo che in questa fase alcuni membri presenti alla riunione potrebbero sentir parlare del progetto per la prima volta e quindi anche in questo caso serve rinfrescare tutto.
\subsubsection{Kick-off meeting Agenda}
L’agenda dovrebbe avere la seguente struttura:
\begin{itemize}
	\item \textbf{Introduzione}
	\item \textbf{Presentazione dello sponsor al team di progetto}: lo sponsor spiega le ragioni, come è nato il progetto, perché è importante.
	\item \textbf{Presentazione degli aspetti rilevanti del progetto da parte dello sponsor}
	\item \textbf{Presentazione del progetto (committente)}: qua potrebbe apparire anche il responsabile del reparto o il CEO che viene a responsabilizzare il team.
	\item \textbf{Presentazione del progetto (project manager)}: parla degli aspetti organizzativi, delle milestone, delle timeline, chi lavora, come e quando. In pratica fornisce un riassunto del piano.
	\item \textbf{Presentazione dei membri del team di progetto}
	\item \textbf{Definire il Project Definition Statement (preparato durante lo scoping e il planning)}: In realtà lo abbiamo già preparato. L'importante è averlo. Le informazioni devono essere presenti e rintracciabili.
	\item \textbf{Stabilire le regole operative del team}: In questa fase molti componenti non servono più e possono abbandonare la riunione. In questa fase ci si mette d'accordo su aspetti che riguardano il modo in cui il team lavora
	\begin{itemize}
		\item \textbf{Problem solving}: esistono informazione chiave su come affrontare un problema. Una volta identificato si definisce come il team lo affronta, preventivamente organizzando riunioni. Il punto è organizzare un processo, in modo che quando compare un problema si seguono indicazioni predefinite per poterlo risolvere efficacemente.
		\item \textbf{Decision making}: si definisce il modo in cui si prendono decisioni.
		\item \textbf{Conflict resolution}: se c'è un problema su determinate specifiche si definisce chi ha la responsabilità.
		\item \textbf{Consensus building}: decidere come lavorare insieme per creare consenso sulle soluzioni.
		\item \textbf{Brainstorming}: si definisce come si organizza il processo di brainstorming.
		\item \textbf{Team meetings}: quali meeting faremo, quando e come.
	\end{itemize}
	\item \textbf{Integrare nella schedula le disponibilità dei membri del team}: si fanno i conti con le agende dei membri del team e si verifica se sono necessarie modifiche alla schedula. Ogni giorno si combatterà per cercare di mantenere la schedula in una condizione accettabile e si cambieranno i piani se necessario.
	\item \textbf{Identificare e scrivere i Work Packages}: Un work package è un insieme di attività che messe insieme hanno un certo significato importante (ad esempio l'insieme di attività che permette di definire una milestone, quelle ad alto rischio, etc.) .
\end{itemize}

\subsubsection{Project Definition Statement}
Come abbiamo già detto, il PDS è una versione estesa del POS a uso del team di progetto. In particolare ha i seguenti impieghi:
\begin{itemize}
	\item \textbf{E’ una buona base per la pianificazione del progetto}: quidni anche per la ri-pianificazione del progetto se dobbiamo cambiare qualcosa;
	\item \textbf{Chiarisce al team in cosa consiste il progetto}: utile alle persone che non hanno preso parte alle fasi precedenti;
	\item \textbf{E’ un riferimento importante per consentire al team di rimanere focalizzati sulla corretta direzione}: Rispettare lo scope di progetto e l'integrità concettuale grazie a questo documento, che riesce a chiarirmi le idee ogni volta che ho un dubbio;
	\item \textbf{E’ un punto di riferimento per i nuovi membri del team}: chi non ha mai sentito parlare del progetto avrà un'idea studiando questo documento;
	\item \textbf{Consente al team di approfondire ed eventualmente individuare nuovi aspetti importanti per il successo del progetto}: In tutta questa documentazione finiscono anche i criteri di accettazione ed altri parametri che riescono a farmi capire se sto andando bene (definizione del DONE specifica per le user story etc.).
\end{itemize}
\subsection{Assegnazione delle responsabilità}
Delegare e assegnare correttamente le responsabilità è un fattore di successo per la gestione dei progetti.
Si possono seguire diversi schemi, un esempio è la matrice RASCI:
\begin{itemize}
	\item \textbf{Responsible}: Responsabile dell’attività e del suo completamento con successo. Se c'è un problema possiamo chiamare lui. Se qualcosa va storto lui ha la responsabilità di dare risposte sul perché qualcosa non ha funzionato e deve cercare di risolvere. Questa figura di solito viene classificata come obbligatoria;
	\item \textbf{Accountable}: Incaricato dell’approvazione del risultato dell’attività. Il TDD (test driven development) ha rimosso il bisogno di avere a disposizione una persona che gestisce questi aspetti. Se esiste però per esempio un concetto di look and feel o di user experience da valutare una persona è necessaria;
	\item \textbf{Support}: Risorsa assegnata per supportare il responsabile. Utile se il responsabile è molto impegnato in altri progetti;
	\item \textbf{Consulted}: Disponibile per assistere il responsabile;
	\item \textbf{Informed}: Membro che deve essere tenuto informato sullo stato di avanzamento. Può anche essere il cliente stesso che ha fretta.
\end{itemize}
\centeredImage{document/img/rasci.png}{RASCI Matrix}{1.0}

\subsection{Regole operative per il team}
Stabilire le regole operative per il team di progetto è fondamentale per concordare i processi che devono essere eseguiti.\newline
Gli aspetti che devono essere regolamentati sono i seguenti:
\begin{itemize}
	\item \textbf{Problem solving}: Il processo di problem solving può essere strutturato in diversi modi. Un esempio sono “i cinque passi” suggeriti da Daniel Couger:
	\begin{enumerate}
		\item \textbf{Definire il problema e il “proprietario” (owner)}: l'owner può aiutarci ad ottenere una descrizione accurata del problema da risolvere e quindi anche strategie per affrontalo;
		\item \textbf{Raccogliere i dati rilevanti e analizzare le cause}: inizialmente si conosce il problema ma non sappiamo perché è avvenuto (es: si è interrotto un servizio). A volte è anche possibile riprodurre il problema e questo semplifica notevolmente il lavoro;
		\item \textbf{Generare delle idee}: come risolviamo il problema? Un problema può essere anche del tipo "sono un utente e non capisco la UI". Un problema di questo tipo può avere molte più "idee" di problemi legati al malfunzionamento di sistemi informatici;
		\item \textbf{Valutare e assegnare una priorità alle idee}: tra le opzioni trovate qual è la soluzione migliore? Anche in questo caso si fa dibattito e si sceglie un'opzione;
		\item \textbf{Sviluppare un piano d’azione}: si crea a tutti gli effetti un'attività per risolvere il problema.
	\end{enumerate}
	Segue l’implementazione della soluzione e la verifica del risultato ottenuto.
	Se la soluzione non è soddisfacente, il processo può essere ripetuto. Risolvere problemi può consumare molto tempo
	\item \textbf{Decision making}: questo processo è pervasivo. Le decisioni possono essere prese in molteplici ambiti:
	\begin{itemize}
		\item Cosa deve essere fatto e dove? (Scope)
		\item Perché dovrebbe essere fatto? (Justification)
		\item Quanto bene deve essere fatto? (Quality)
		\item Quando è richiesto? In quale sequenza? (Schedule)
		\item Quanto costerà? (Budget/Cost)
		\item Quali sono le incognite? (Risk)
		\item Chi dovrebbe fare il lavoro? (Human Resource)
		\item Come dovrebbero essere organizzate le persone all’interno del team? (Communication/Interpersonal Skills)
		\item Come sapremo? (Information Dissemination/Communication)
	\end{itemize}
	Il processo di decision making può seguire principalmente tre approcci alternativi:
	\begin{itemize}
		\item \textbf{Direttivo}: la persona che detiene la responsabilità/autorità prende tutte le decisioni per tutti i membri del team. Per esempio, potrebbe essere il project manager per decisioni riguardanti il progetto o il “task manager” per i singoli task.
		\item \textbf{Participativa/Collaborativa}: ogni membro del team partecipa al processo decisionale (decision-making). La sinergia tra i membri del team consente di individuare la migliore decisione e aumenta il loro coinvolgimento.
		\item \textbf{Consultativa}: costituisce la soluzione di compromesso, in cui la persona che detiene la responsabilità/autorità prende la decisione, ma dopo aver raccolto informazioni e idee dai membri del team.
	\end{itemize}
	\item \textbf{Conflict resolution}: Le tipologie di persone con cui possiamo avere a che fare possono essere molto assertive o non assertive, molto cooperative e poco cooperative.
	\begin{itemize}
		\item Il top sarebbe avere persone sia cooperative che assertive, queste sono denominate \textbf{collaborating}.
		\item I \textbf{competing} sono persone molto assertive e poco cooperative tendono ad imporre la loro idea con forza rigettando quella degli altri.
		\item Gli \textbf{accomodating} sono persone che fanno ciò che gli viene chiesto, mettendo poco in discussione soluzioni.
		\item Gli \textbf{avoiding} sono il caso peggiore, solitamente la loro presenza è dannosa o inutile.
	\end{itemize}
	\centeredImage{document/img/conflict.png}{Conflict resolution}{0.6}
	\item \textbf{Consensus building}: Questo è una specie di warning. Il consenso totale da parte del team non implica necessariamente che la decisione presa sia quella giusta. Spendere del tempo in più per indagare sulle ragioni del consenso può portare a capire l'errore.
	\item \textbf{Brainstorming}: Si ricorre al “brainstorming” per individuare delle soluzioni che difficilmente un singolo può identificare. Lavorando in gruppo si possono identificate più soluzioni e scegliere la migliore. Aspetti fondamentali del brainstorming:
	\begin{itemize}
		\item \textbf{Riunire chi ha conoscenza dell’ambito in cui si riscontra il problema}: chi conosce l'ambito cerca di buttare sul tavolo delle idee iniziali in modo da far nascere una specie di dibattito;
		\item \textbf{Mettere sul tavolo tutte le idee}: si enumerano tutte le possibili alternative;
		\item \textbf{Continuare il processo finché non si esauriscono le idee};
		\item \textbf{Discutere tutte le idee che sono emerse};
		\item \textbf{Durante il processo di discussione la soluzione inizia a emergere};
		\item \textbf{E’ necessario testare tutte le idee con estrema apertura mentale}: capire le ragioni dell'altro può portare allo sviluppo di ulteriori idee.
	\end{itemize}
	\item \textbf{Team meetings}: Le riunioni tra i membri del team sono uno strumento indispensabile nella fase di esecuzione. Nella fase di Launching i meeting non sono strutturati allo stesso modo delle fasi precedenti dato che una riunione può essere necessaria per molteplici motivi anche ciò ch va discusso deve essere ben definito. Per la loro organizzazione bisogna rispondere alle seguenti domande:
	\begin{itemize}
		\item \textbf{Qual è lo scopo della riunione?};
		\item \textbf{Quanto spesso bisogna riunirsi? Per quanto tempo?}: Per esempio la review di fine sprint in scrum devono essere ben definite e i team members devono approvare;
		\item \textbf{Chi dovrebbe partecipare?}: dipende dal tipo di riunione, dagli scopi etc. Spesso le aziende hanno già policy aziendali che definiscono chi deve partecipare e a quali riunioni.
		\item \textbf{C’è bisogno di un’agenda?}: a volte non è necessaria.
		\item \textbf{Deve essere redatto un verbale? Chi lo scrive? Chi lo deve consultare?}: Lo stand up meeting per esempio non ha bisogno di un verbale, ma in altri casi è la normalità stendere un riassunto di ciò che è stato detto.
	\end{itemize}
\end{itemize}

\subsubsection{Linee guida per la gestione delle riunioni}
\paragraph{Prima del meeting}
\begin{itemize}
	\item \textbf{Ci si deve chiedere se la riunione è necessaria}: a volte il problema che si vuole discutere non necessita di una riunione
	\begin{info}[Esempio]
		Per risolvere un bug spesso non è necessaria una riunione e ci si affida al responsabilità dell'attività problematica. Se però il bug riscontrato non dipende dal lavoro svolto dal team guidato dal responsabile probabilmente serve una riunione perché potrebbe essere causato da problemi di integrazione con altre attività.
	\end{info}
	\item \textbf{Determinare lo scopo della riunione}: comporta anche cambiamenti su CHI deve partecipare alla riunione;
	\item \textbf{Stabilire le regole di base per la discussione}: Il facilitatore dovrà gestire i tempi in modo da poter convergere ad un risultato. Le riunioni tendono infatti ad essere dispersive;
	\item \textbf{Determinare chi davvero deve partecipare e invitare solo questi membri (inutile far perdere tempo a chi non è necessario)};
	\item \textbf{Preparare delle note e la presentazione}: Utile portare materiale per condurre la riunione;
	\item \textbf{Stabilire la data e la durata della riunione, che dovranno essere rispettate con la massima puntualità}: utile a tutti per poter organizzare gli altri impegni.
\end{itemize}

\paragraph{Durante il meeting}
\begin{itemize}
	\item \textbf{Ribadire la durata della riunione e rispettarla (la durata deve essere specificata quando avviene la convocazione)};
	\item \textbf{Identificare obiettivi specifici}: Ci si ricorda a cosa serve la riunione e si fornisce una lista dei risultati che si vogliono ottenere;
	\item \textbf{Raccogliere gli input dai partecipanti}: si mettono sul tavolo le idee dei partecipanti dato tutit il materiale loro fornito;
	\item \textbf{Non fossilizzarsi su alcune posizioni}: cercare sempre di progredire nella discussione. Il facilitatore in questo caso ha un ruolo fondamentale nel cercare di convergere;
	\item \textbf{Aiutarsi con ausili visivi}: Una presentazione che permette di far capire il punto ai partecipanti può rendere molto veloce l'esecuzione della riunione;
	\item \textbf{Periodicamente riassumere i risultati raggiunti dalla discussione in termini di consenso acquisito o disaccordo ancora non risolto};
	\item \textbf{Assegnare a ciascun membro del team delle possibili azioni}.
\end{itemize}

\paragraph{Dopo il meeting}
\begin{itemize}
	\item \textbf{Fissare quando e dove si farà la prossima riunione e stabilire la possibile agenda};
	\item \textbf{Quando viene fissata la data e il luogo della prossima riunione, stabilire anche chi deve partecipare, specificando il ruolo (sia nella riunione che nel progetto)};
	\item \textbf{Riportare gli argomenti in agenda effettivamente discussi}: Una volta passato il tempo della riunione, se non sono stati conclusi i punti si possono riprendere in una riunione futura;
	\item \textbf{Riportare le decisioni prese o quelle ancora in sospeso perché richiedono ulteriori approfondimenti}: sarebbe opportuno anche comunicarle a chi deve aver conoscenza delle decisioni;
	\item \textbf{Definire le azioni da intraprendere e i responsabili che dovranno garantire l’esecuzione e riportare l’esito al team nella prossima riunione}.
\end{itemize}

\subsubsection{Tipologie di Project Meetings}
Le tre tipologie base di riunioni sono:
\begin{itemize}
	\item \textbf{Daily status meetings}: non tutti li fanno. Le riunioni giornaliere servono per monitorare lo stato di avanzamento dei task del progetto.
	\begin{itemize}
		\item Devono avere una breve durata: 15 minuti;
		\item Si può fare senza sedersi in una sala riunioni;
		\item Si parla solo dei tasks aperti su cui si sta lavorando e che ancora non sono stati completati;
		\item Ogni responsabile di un task riporta lo stato di avanzamento:
		\begin{itemize}
			\item “Sono in schedula”;
			\item “Sono in ritardo di x ore, ma prevedo di rientrare in schedula entro y giorni” (possibilmente 1-2 giorni);
			\item “Sono in ritardo di x ore e ho bisogno di aiuto per recuperare”;
			\item “Sono in anticipo di x ore e posso aiutare chi ha bisogno”.
		\end{itemize}
	\end{itemize}
	\item \textbf{Problem resolution meetings}: si effettuano solo nel momento del bisogno. I problemi non devono essere risolti nelle riunioni giornaliere, ma in riunioni specifiche. In queste riunioni per la risoluzione di un problema:
	\begin{itemize}
		\item Partecipa solo chi è coinvolto nel problema;
		\item Si determina chi è il “proprietario” (owner) del problema;
		\item Si deve identificare la soluzione;
		\item Si deve determinare come e quando potremo stabilire che il problema è risolto.
	\end{itemize}
	\item \textbf{Project review meetings}: si effettuano perché pianificati. Queste riunioni sono eventi formali, che si tengono in concomitanza con il raggiungimento di una milestone. Durante questi eventi:
	\begin{itemize}
		\item Si svolge la presentazione dello stato del progetto in corrispondenza della milestone e si svolge una sua revisione critica;
		\item Partecipano il project manager, il senior management, il committente, lo sponsor, gli stakeholders e alcuni tecnici esperti dell’ambito di progetto oggetto della milestone;
		\item Si presentano le performance fornite dal progetto fino a quel momento;
		\item Si possono proporre anche delle azioni correttive.
	\end{itemize}
\end{itemize}
\paragraph{Team War Room}
Potrebbe aver senso avere delle stanze fatte apposta per i team per prendere decisioni. La “War Room” è uno spazio dedicato alle riunioni necessarie durante il progetto. La stanza deve fornire tutte le facilities necessarie.
\subsection{Gestione dei cambiamenti di scope}
\centeredImage{document/img/scopechange.png}{Processo tipico per la gestione del cambiamento di Scope}{0.4}
In un qualsiasi momento un qualsiasi stakeholder può chiedere di apportare cambiamenti. Bisogna a questo punto capire la richiesta, la quale deve essere compresa. Esistono template per accettare la proposta di cambiamento, in modo da rendere più semplice formalizzare la richiesta di cambiamento e avere a disposizione tutte le informazioni necessarie. Se nonostante i template la richiesta non è ancora chiara può esserci una retroazione in cui viene richiesta una spiegazione migliore.
\centeredImage{document/img/scopechangeform.png}{Esempio di form per le richieste di cambiamento dello scope}{0.3}
Una volta capita la richiesta è possibile:
\begin{enumerate}
	\item rigettarla (non significa necessariamente non parlarne mai più);
	\item ritenere che possa aver senso, ma che non impatta a sufficienza sul business value;
	\item passare alla fase successiva, facendo uno studio sull'impatto;
\end{enumerate}
Da una parte c'è l'impatto in termini di valore restituito, dall'altra quanto alti sono i costi, quanto tempo servirà per implementare il cambiamento. Si effettua poi una fase di review dell'impatto. Anche in questo caso si parla con le figure che riguardano il cambiamento dello scope.
\begin{info}
	Se per esempio il cambiamento dello scope riguarda solo l'architettura interna del sistema il cliente può anche non essere coinvolto nella fase di review. In questo caso basta contattare anche solo il senior manager dell'azienda.\newline
	\noindent Se invece l'impatto è sul cliente bisogna spiegare le ragioni e quali implicazioni ha il cambiamento di scope. Inoltre va messo al corrente che il cambio di scope ha un costo.
\end{info}
Il cambio dello scope implica anche un cambio nei piani e nella schedula del progetto. Potrebbero servire del tempo e delle risorse in più. Il progetto intero può subire ritardi, oppure è possibile sfruttare gli slack per cercare di mitigarlo. In pratica è necessario un vero e proprio processo di scoping all'interno della fase di esecuzione, in modo da poter avere un nuovo piano che sarà poi possibile seguire.

\subsubsection{Project Impact Statement}
Documento preparato dal project manager, con la collaborazione del team, per definire l’impatto del cambiamento di scope sul progetto.
Deve rispondere alle seguenti domande:
\begin{itemize}
	\item \textbf{Qual è il beneficio atteso dal cambiamento?}: ossia, qual è il business value;
	\item \textbf{Come impatterà il cambiamento sui costi?}: l'impatto può essere anche nullo, magari togliendo qualcosa e aggiungendo una tecnologia che ha pressoché lo stesso costo. In alcuni casi il costo potrebbe addirittura diminuire;
	\item \textbf{Come impatterà il cambiamento sulla schedula del progetto?}: l'introduzione del cambiamento ha un impatto forte sulla deadline del progetto?;
	\item \textbf{Come impatterà il cambiamento sulla qualità della soluzione?}: molti cambiamenti sono molto vicini a delle patch piuttosto che a veri e proprio cambiamenti. Questo può essere dovuto anche all'inesperienza che ha portato a non identificare nel modo scorretto lo scope nelle fasi iniziali del progetto. In questi casi il cambiamento spesso viene accolto "per forza", ma bisogna stare attenti perché la stabilità della soluzione potrebbe risentirne pesantemente;
	\item \textbf{Come impatterà il cambiamento sull’allocazione delle risorse?}: ad esempio la rimozione di una certa tecnologia può implicare che non si ha più bisogno di una certa figura all'interno del progetto, magari proprio l'esperto della tecnologia in questione;
	\item \textbf{Il cambiamento può essere rimandato a un successivo stadio di progetto o a una prossima versione della soluzione?}: il cambiamento dello scope può portare ad ulteriori cambiamenti e a problemi di integrazione;
	\item \textbf{Il progetto è in uno stato in cui è alto il rischio che il cambiamento richiesto destabilizzi la soluzione?}: quando vengono inseriti nuovi elementi all'interno del progetto bisogna valutare attentamente l'impatto del rischio.
\end{itemize}
La definizione dell’impatto del cambiamento può avere i seguenti esiti:
\begin{itemize}
	\item \textbf{Il cambiamento può essere applicato entro le risorse e i tempi previsti per il progetto}: l'unico problema da risolvere in questo caso è stata rendere compatibile il cambiamento con gli obiettivi di progetto. Se il cambiamento è utile, ha un enorme valore, ma non è compatibile con gli obiettivi in generale si potrebbe pensare che gli obiettivi sono stati valutati male e quindi vanno cambiati. Partire convinti di sapere tutto e scoprire che il cliente ha una nuova richiesta è normale;
	\item \textbf{Il cambiamento può essere applicato, ma richiederà un’estensione della schedula}: il cliente deve fornire l'ok per il tempo in più che serve;
	\item \textbf{Il cambiamento può essere applicato entro la schedula prevista, ma sono richieste ulteriori risorse}: bisogna verificare se nello staff sono disponibili le risorse necessarie per riuscire a finire comunque in tempo;
	\item \textbf{Il cambiamento può essere applicato, ma sono richieste ulteriori risorse e un’estensione della schedula}: è il caso in cui convergono gli ultimi due punti sopra citati;
	\item \textbf{Il cambiamento può essere applicato utilizzando una strategia “multiple release”, stabilendo le priorità dei deliverables e le date di rilascio}: è possibile suddividere le consegne del progetto, consegnando versioni del software con i cambiamenti desiderati in futuro;
	\item \textbf{Il cambiamento non può essere applicato senza modifiche sostanziali del progetto}.
\end{itemize}

\subsubsection{Aiutare il cliente a capire il cambiamento di Scope}
Il cliente potrebbe spesso pensare che un cambiamento possa risultare facile e poco costoso, quando in pratica questo può richiedere cambiamenti notevoli nel progetto.\newline
\noindent Il cambiamento può essere dovuto a cambiamenti esterni (del mercato, delle tecnologie, etc.) oppure alla scoperta di nuovi bisogni/requisiti durante il progetto, che non erano emersi nella fase di scoping.
Le richieste di cambiamento dello scope devono essere sempre messe in conto. Il processo di cambiamento dello scope dovrebbe sempre essere definito preventivamente. Il team di progetto guidato dal project manager analizzerà l’impatto della richiesta di cambiamento sull’intero progetto.\newline
Il cliente dovrebbe sempre essere messo al corrente del cambiamento che c'è stato, a meno che non sia possibile sistemare la schedula in modo che il cambiamento risulti trasparente dal suo punto di vista (vale ovviamente anche per i costi).\newline
\noindent Il project impact statement definirà tutte le possibili alternative per soddisfare la richiesta di cambiamento. Assieme al committente si sceglierà una delle alternative individuate nel project impact statement. Infine il project manager ristrutturerà il piano del progetto in conseguenza della scelta effettuata e informerà il committente.
\subsubsection{Gestione di una riserva}
Una percentuale della durata del progetto (tipicamente del 5-10\%) è “accantonata” come tempo aggiuntivo per far fronte alle contingenze (time contingency) relative all’elaborazione e all’eventuale conseguente accoglimento delle richieste di cambiamento dello scope.\newline
\noindent Le richieste di cambiamento possono essere processate ogni volta che emergono oppure ad ogni “ciclo” del progetto. La gestione della riserva può essere aiutata dall’utilizzo di una sorta di “Scope Bank” per “depositare” i requisiti (funzioni, feature, etc.) non ancora integrati nella soluzione, a cui si può assegnare anche una priorità.
La situazione ideale sarebbe indovinare i costi del progetto, determinare un margine, ed essere molto bravi a rimanere nei costi iniziali senza la necessità di dover attingere dal margine.
\subsection{Gestione delle comunicazioni}
Per gestire le comunicazioni bisogna definire:
\begin{itemize}
	\item \textbf{Il “timing”}: quando comunichiamo;
	\item \textbf{I contenuti}: di cosa parliamo. In alcune comunicazioni ricorrenti spesso basta solo fornire dei valori (ad esempio ogni mese bisogna comunicare delle spese ad un ufficio);
	\item \textbf{Scegliere i canali che consentono un’efficiente comunicazione}: ogni metodo di comunicazione ha i suoi pro e i suoi contro. Ci sono situazioni in cui è meglio prediligere certi canali rispetto ad altri.
\end{itemize}
Ricordiamo che il processo delle comunicazioni parte con la generazione dell'idea, che viene poi codificata in un qualche modo. Il messaggio viene trasmesso attraverso il canale scelto. Ci aspettiamo che il ricevente decodifichi il messaggio. Il ricevente infine fornisce un feedback.
\centeredImage{document/img/comproc.png}{Processo di comunicazione}{0.4}

\subsubsection{Classificazione delle comunicazioni}
\paragraph{One-to-one}
\begin{itemize}
	\item Conversazioni (di persona e al telefono);
	\item Riunioni;
\end{itemize}
\paragraph{Elettronica}
\begin{itemize}
	\item E-mails;
	\item Websites;
	\item Databases;
\end{itemize}
\paragraph{Scritta}
\begin{itemize}
	\item Memos;
	\item Lettere;
	\item Documenti;
	\item Reports;
\end{itemize}
Il problema principale è cercare di catalogare tutte le informazioni in modo che siano rintracciabili, in modo da capire lo stato corrente di una determinata situazione in un tempo accettabile.

\subsection{Gestione ed assegnamento delle risorse}
Nella realtà di tutti i giorni, le risorse disponibili possono anche essere impiegate in altri progetti, pertanto è facile che qualche risorsa soffra della cosiddetta “over-allocation”.
Spesso per esempio non tutto il personale è disponibile per tutta la durata del progetto (ferie, malattia, etc.).
Il project manager cerca sempre di sistemare le risorse e la schedula per minimizzare ritardi e costi aggiuntivi.
Per gestire al meglio le risorse disponibili, quando si presentano delle criticità si possono:
\begin{itemize}
	\item \textbf{Utilizzare gli “slack” disponibili};
	\item \textbf{Far slittare la data di fine del progetto};
	\item \textbf{Ricorrere allo straordinario}: comunque problematico perché la produttività non resta costante, va retribuito e spesso non è comunque sufficiente per risolvere situazioni critiche.
\end{itemize}

\subsection{Metodi alternativi per schedulare i task}
Per semplificare il problema dell’assegnamento delle risorse si può ricorrere a metodi alternativi per pianificare i task. Tra le possibili azioni possiamo citare le seguenti:
\begin{itemize}
	\item \textbf{I task possono essere decomposti ulteriormente};
	\item \textbf{Allungare la durata dei task (nota bene: solo la durata)}: può essere utile per avere flessibilità nella gestione delle risorse. Se l'effort rimane uguale non stiamo aumentando i costi, stiamo solo aumentando il tempo che abbiamo. Nella schedula finale vedo una diminuzione di slack, che può essere un contro ma anche un pro perché scelgo meglio come utilizzare i "tempi morti";
	\item \textbf{Assegnare delle altre risorse in sostituzione di quelle non disponibili o non sufficientemente disponibili}.
\end{itemize}

\subsection{Scrivere i work packages}
Un work package è una breve descrizione che specifica come i task che lo compongono saranno completati. I task inclusi in un work package dovrebbero avere delle relazioni tra loro che ne motivano una gestione coordinata (i motivi possono essere variegati). Spesso i task scelti vengono legati insieme per fare in modo di avere un sistema di reporting che li riguardano, in modo da avere per loro un occhio di riguardo (anche attraverso KPI).
\begin{itemize}
	\item Permettono di aver maggior controllo sul progetto e il rispetto della schedula.
	\item Povrebbero essere scritti per:
	\begin{itemize}
		\item I task nel critical path;
		\item I task ad alto rischio;
		\item I task che richiedono risorse che sono scarse;
		\item I task che hanno un’elevata varianza nella durata;
		\item I task che riguardano il raggiungimento di una milestone (uno degli usi più frequenti);
		\item Sostituire delle risorse assegnate ai task.
	\end{itemize}
\end{itemize}
Un'idea per la gestione dei work package può essere assegnare loro un nome ed un responsabile.
\centeredImage{document/img/workpackagesheet.png}{Work Package Assignment Sheet}{0.5}
\noindent Ogni Work Package possiede quindi una lista di task e tutte le caratteristiche che lo distinguono (ID, nome, responsabile etc.).
