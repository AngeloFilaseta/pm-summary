\section{Linee Guida per il Progetto}
\subsection{In cosa consiste il progetto}
L'esame sarà un orale in cui ci saranno sia domande di teoria che la discussione di un elaborato, il quale consiste nella simulazione di attività di "gestione di un progetto di sviluppo software" reale o di fantasia.\newline
Gli elaborati devono prevedere due sezioni:
\begin{itemize}
	\item \textbf{Descrizione dell'approccio utilizzato};
	\item \textbf{Documentazione di progetto};
\end{itemize}
Non bisogna scrivere software, bisogna solo dimostrare la capacità di gestire un progetto. Prima è necessario concordare con il docente il topic a grandi linee.
\subsection{Quali sono i deliverables}
I deliverables sono i seguenti:
\begin{itemize}
	\item \textbf{Descrizione dell'approccio utilizzato}: Si tratta di un documento in cui deve essere descritto come si è scelto di gestire il progetto, giustificando ogni scelta effettuata. L'organizzazione del documento è a discrezione dello studente, ma ci si aspetta:
	\begin{itemize}
		\item Scoping/Initiating;
		\item Planning;
		\item Launching/Execution;
		\item Monitoring/Controlling;
		\item Closing;
	\end{itemize}
	Per i meeting previsti si chiede di documentare l'ordine del giorno, i partecipanti e una sintesi dell'ipotetico svolgimento.
	\item \textbf{Documentazione di progetto}; L'insieme dei documenti di progetto che lo studente ha deciso di utilizzare (POS, RBS, WBS, PDS, Analisi dei rischi, Gantt, etc.).\newline
	Si incoraggia l'utilizzo di software di supporto (MS Project etc.). Si possono utilizzare approcci alternativi.
\end{itemize}
\subsection{Come sarà valutato}
NON si valutano le tecnologie utilizzate, gli approcci architetturali se non ciò che riguarda la gestione di progetto. Si tratta di mostrare capacità critiche.
