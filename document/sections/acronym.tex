\section{Acronimi}

\begin{tabularx}{\textwidth}{|c|c|X|}
	\hline			
	APM & Agile Project Management & L'approccio che segue le linee guide dell'AGILE Manifesto nel PM.\\
	\hline
	CoS & Condition of Satisfaction & Le condizioni necessarie da soddisfare per portare a termine un progetto. \\
	\hline
	MPx & Emertxe Project Management & 
	Una soluzione che viene utilizzata quando si hanno le idee chiare sulla soluzione da adottare ma non sullo scope.\\
	\hline  
	KPI & Key Process Indicator & Un indicatore che fornisce un'indicazione quantitativa su quanto bene funziona un processo. \\
	\hline  
	PM & Project Management & Secondo PMBOK è l'applicazione di conoscenze, capacità, strumenti, tecniche alle attività di progetto per soddisfare i requisiti.\\
	\hline		
	PMBOK & Project Management Body of Knowledge & La guida che fornisce delle buone linee guida per la gestione di progetti.\\
	\hline  
	PMLC & Project management Life Cycle & Il ciclo di vita è tutto ciò che accade da quando inizia e finisce un progetto.\\
	\hline
	POS & Project Overview Statement & 3 \\
	\hline
	RFI & Request for Information & Un documento che si può inviare ai vendors per avere chiarimenti ulteriori sul target dell'RFP. \\
	\hline
	RFP & Request for Proposal & Documento da inviare ai fornitori per chiedere se sono interessati alla nostra richiesta. \\
	\hline
	RBS & Requirements Breakdown Structure & Una struttura ad albero per la categorizzazione e la suddivisione dei requisiti in attività. \\
	\hline
	TPM & Traditional Project Management & L'approccio che segue il modello a cascata tradizionale nel PM. \\
	\hline  
	WBS & Work Breakdown Structure & Struttura ad albero che invece viene utilizzata per definire le attività di progetto.\\
	\hline  
	xPM & Extreme Project Management &  Una soluzione che viene utilizzata quando non si hanno le idee chiare ne sul goal ne sulla soluzione.\\
	\hline
\end{tabularx}
