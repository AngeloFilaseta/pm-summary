\section{Scoping Process Group}
\subsection{Tools, templates e processi per definire lo scope di un progetto}
Per definire lo scope di progetto è possibile utilizzare tantissimi strumenti, templates e processi, ce ne sono veramente tanti, anche la riunione è considerabile uno strumento, eccone alcune:
\begin{itemize}
	\item Conditions of Satisfaction;
	\item Project Scoping Meeting (riunioni);
	\item Raccolta dei requisiti ( Requirements Gathering);
	\item Definizione dei “ Processi Aziendali ” / Workflow;
	\item Diagrammi di processi aziendali (Diagramming Business Processes);
	\item Prototipazione (Prototyping);
	\item Convalida dei casi aziendali (Validating Business Cases);
	\item Project Overview Statement (POS);
	\item Approvazione da parte del senior management del POS e “nulla osta” a procedere alla pianificazione;
\end{itemize}
\subsection{Gestire le aspettative del committente}
Il problema di base che bisogna risolvere è trovare il giusto equilibrio tra ciò di cui il cliente necessita e ciò che il cliente desidera. Perché non è una buona idea assecondare il cliente e dargli tutto ciò che desidera? Come mai dobbiamo fermarci e mettere dei paletti?\newline
Il cliente a fine progetto potrebbe non rimanere soddisfatto, perché il desiderio potrebbe non portare a business value. Focalizzarsi troppo sui desideri porta a oscurare i bisogni e viceversa. In questa fase è anche possibile per dei consulenti aiutarci. Il cliente poi avrà delle aspettative. Bisogna assicurarsi che il cliente abbia capito cosa verrà fatto. Le aspettative devono essere allineate con ciò che pensiamo di fare. Il committente deve avere un ruolo attivo durante tutta la fase di scoping. Più coinvolgiamo il committente meglio è. Ogni volta che c'è un dubbio il committente va interpellato.
\centeredImage{document/img/scopingmeeting.PNG}{Piano per redigere il POS secondo Wysocki}{0.5}
\noindent Critichiamo insieme questa figura.
\begin{itemize}
	\item Conduciamo la Condition of Satisfaction, capiamo quali sono le condizioni per soddisfare il progetto;
	\item Definiamo i requisiti;
	\item Ci chiediamo se redigere il POS, in caso positivo lo sottoponiamo e basta. In caso negativo allora creiamo la Requirements Breakdown Structure, verifichiamo la completezza dei requisiti, classifichiamo il progetto in base ai requisiti, determiniamo il miglior modello di gestione del progetto, scriviamo il POS e poi lo sottoponiamo.
\end{itemize}
Cosa manca in questa figura? Il problema è che il processo è in realtà molto più complicato di come viene riportato in questa figura (molto lineare).
\centeredImage{document/img/scopingpos.PNG}{Piano per redigere il POS}{0.5}
\noindent Utilizzando questo approccio dividiamo in due macroprocessi:
\begin{itemize}
	\item Nel primo cerchiamo di elencare i requisiti creando ed aggiornando l'RBS mano a mano che abbiamo nuove iterazioni.
	\item Una volta capiti i requisiti possiamo capire qual è il modello migliore e più adatto per il progetto. Anche in questo caso però sono necessarie varie iterazioni.
\end{itemize}
Questo approccio però comporta dei rischi, perché può portarci alla non convergenza, all'aggiornamento continuo dei requisiti. Una buona pratica è fissare i requisiti sicuramente certi per mitigare i rischi il più possibile.
\subsection{Conditions of Satisfaction (CoS)}
Torniamo su questa immagine:
\centeredImage{document/img/cos.PNG}{Condition of Satisfaction 2 - La vendetta}{0.5}
\noindent Condition of Satisfaction e requisiti sono due cose diverse. CoS serve per definire quali sono i requisiti da rispettare durante tutto il ciclo di vita di progetto per raggiungere il risultato finale. Non coincidono con i goal e gli obiettivi del progetto, ma possono indicare elementi aggiuntivi che devono essere rispettate:
\begin{itemize}
	\item rispetto del budget;
	\item completamente di alcuni componenti entro una certa data;
	\item adesione ad uno standard ben preciso;
	\item molto altro...
\end{itemize}
\centeredImage{document/img/cosexample.PNG}{Condition of Satisfaction - Esempio}{0.7}
Le \textbf{Conditions Of Satisfaction} determinano il "cosa" deve essere il risultato del progetto, ovvero quali sono gli obiettivi. L'esempio classico in Scrum è il Product Owner che deve gestire le CoS per capire bene cosa vuole l'utente.
L'\textbf{Acceptance Criteria} riguarda i fattori che permettono di determinare se il risultato del progetto può essere accettato. Questi fattori sono molto utili anche per definire quali sono i test che il software deve sicuramente passare. Se l'utente trova una casistica non testata alla consegna o non è stata prima definita o ci siamo scordati noi di implementarla. Questi criteri possono far parte dello scoping, ma vanno definiti successivamente, per esempio quando definiamo anche il "come" procedere.
\subsection{Project scoping meeting}
Tutto ciò che riguarda lo scope viene deciso durante le riunioni. In questo caso parliamo di \textbf{Project Scoping Meeting}, che servono appunto per decidere lo scope.
\subsubsection{Output dei Project Scoping Meeting}
Una delle cose da fare a priori è specificare qual è lo scopo della riunione. L'output delle riunioni possono essere l'RBS o il POS. Non usciranno questi documenti già dalla prima riunione, si tratta di un processo che richiede più iterazioni che vanno via via a dettagliare sempre maggiormente il quadro.
\subsubsection{Chi deve partecipare?}
Tutto il \textbf{team} deve partecipare alle riunioni, sicuramente non bisogna scordarsi degli \textbf{stakeholders}. Non devono necessariamente partecipare a ogni riunione ma in quelle in cui il loro parere è core è molto importante che loro siano presenti. Una figura importante è il \textbf{facilitatore}, ovvero per esempio un esperto del dominio che rende estremamente più chiaro capire come approcciare certe soluzioni. Si tratta della figura che aiuta a mantenere l'ordine.\newline
Il \textbf{"tecnografo"} è la figura che accumula i dati e li mette a disposizione di tutti anche grazie a qualche framework.
\subsubsection{Agenda (esempio)}
Un tipico piano per una riunione può essere quello che segue. In molti casi è importante inviare uno straccio di Agenda a tutti i partecipanti perché se serve del materiale in anticipo si ha il tempo di prepararlo.
\begin{itemize}
	\item \textbf{Introduzione}: Non tutti sanno di cosa tratta il progetto. L'introduzione serve per inquadrare la situazione. Non è mai una perdita di tempo. Questa parte ci aiuta anche a capire chi è il committente e perché ha bisogno di noi.
	\item \textbf{Scopo del Meeting}: In questo caso può parlare molto il facilitatore che spiega più dettagliatamente lo scopo della riunione.
	\item \textbf{Discussione delle CoS}.
	\item \textbf{Descrizione dello stato corrente}: Entriamo più del merito del perché ci serve il nostro progetto. Ci chiediamo come funziona ora l'azienda e quali sono i bisogni, cosa bisogna migliorare, cosa bisogna rimuovere o aggiornare.
	\item \textbf{Descrizione del problema o della business opportunity}: Si descrive ancora meglio il valore che può portare il progetto rispetto alla situazione attuale.
	\item \textbf{Descrizione dello stato finale da raggiungere}: Si descrive lo stato in cui si vuole arrivare, ovvero i goal e gli obiettivi ad un basso livello di dettaglio.
	\item \textbf{Definizione dei requisiti e manutenzione}: Sempre ad alto livello si cominciano ad elencare i requisiti.
	\item \textbf{Discussione del "gap" esistente tra lo stato corrente e quello che si vuole raggiungere al termine del progetto}: Si cerca di definire il cambiamento che deve esserci in termini quantitativi (es: da una rottura ogni 100 prodotti dobbiamo arrivare a una ogni 1000). Inizialmente non sarà facile fare stime, ma dipende anche dalla difficoltà e la dimensione del progetto.
	\item \textbf{Scelta dell'approccio per gestire il progetto (PMLC model) che meglio si adatta al "gap" individuato}.
	\item \textbf{Bozza e approvazione del POS}.
	\item \textbf{Aggiornamento a una eventuale riunione successiva}: Alcuni dati potrebbero essere mancanti o assenti in delle riunioni. In questo caso ci si aggiorna meglio nelle riunioni successive, aggiornando i documenti.
\end{itemize}
\subsubsection{Deliverables dello Scoping Meeting}
\begin{itemize}
	\item CoS
	\item Requirements Document (RBS)
	\item Best-fit project management life cycle (PMLC model)
	\item POS
\end{itemize}
\subsection{Requirements Breakdown Structure (RBS)}
Lo scopo è raccogliere tutti i requisiti (Requirements Gathering), partendo da requisiti di alto livello fino ad arrivare a quelli di basso livello, per cui sarà necessario sempre di più essere a contatto con lo stakeholders. Gli strumenti a disposizione per evitare problemi durante la raccolta dei requisiti sono i seguenti:
\begin{itemize}
	\item \textbf{Facilitated Group Session}:  Si tratta di rinuioni più piccole in cui solo alcuni tipi di utenti vengono interpellati perché serve fare domande rapide e mirate.
	\begin{info}[]
		\textbf{Punti di forza:}
		\begin{itemize}
			\item Maggiori dettagli vengono scoperti molto rapidamente;
			\item Un facilitatore imparziale aiuta a risolvere problemi facilmente;
			\item l'approccio funziona bene per i processi iter-funzionali.
		\end{itemize}
		$\;\;$\textbf{Rischi:}
		\begin{itemize}
			\item Un facilitatore non esperto può condurre a risposte negative;
			\item Può essere costoso pianificare ed eseguire la riunione se devono essere presenti persone specifiche.
		\end{itemize}
	\end{info}
	\item \textbf{Interviews}: Le interviste agli utenti finali forniscono informazioni utili, soprattutto per i requisiti a basso livello.
	\begin{info}[]
		\textbf{Punti di forza:}
		\begin{itemize}
			\item Partecipano anche gli end-users;
			\item Descrizione ad alto livello delle funzioni e dei processi forniti. Ovvero possiamo ricostruire alcuni processi che non sono documentati ma che ci servono per portare avanti il progetto.
		\end{itemize}
		$\;\;$\textbf{Rischi:}
		\begin{itemize}
			\item Non costruendo bene l'intervista potremmo non ottenere informazioni utili;
			\item Se l'analista è prevenuto potrebbero essere ignorati i veri bisogni di un committente. L'intervistatore non deve "aver già capito tutto". Non bisogna fare domande solo per confermare un'ipotesi, bisogna farne anche per contraddirla;
			\item Le descrizioni possono differire dalle attività effettive, soprattutto non vengono espressi dettagli importanti.
		\end{itemize}
	\end{info}
	\item \textbf{Observation}: Gli utenti fanno fatica a descrivere le situazioni. Un buon modo per scoprire come funziona un processo è osservarlo.
	\begin{info}[]
		\textbf{Punti di forza:}
		\begin{itemize}
			\item Se è l'analista ad osservare il processo non c'è spazio per errori dovuti alla comunicazione. Le descrizioni saranno generalmente più dettagliate, precise e complete;
			\item Approccio molto utile se le attività sono difficili da descrivere.
		\end{itemize}
		$\;\;$\textbf{Rischi:}
		\begin{itemize}
			\item Osservare e documentare è molto dispendioso, sia in termini di costo che di tempo. Spesso ci sono anche problemi legali nel farlo;
			\item Alcune informazioni possono essere confuse o in contrasto con altre già definite;
			\item Ciò che viene osservato potrebbe essere male interpretato. In questo caso è utile fare domande per capire se ciò che abbiamo visto è anche ciò che abbiamo capito.
		\end{itemize}
	\end{info}
	\item \textbf{Requirements Reuse}: Se si progetta in un ambito in cui si ha già operato in passato probabilmente si hanno già delle idee, del software, delle librerie etc..
	\begin{info}[]
		\textbf{Punti di forza:}
		\begin{itemize}
			\item I requisiti vengono rifiniti molto velocemente;
			\item Le attività di raccolta vengono ridotte;
			\item La soddisfazione del committente può essere rafforzata rispetto alla scorsa esperienza positiva;
			\item Aumento della qualità;
			\item Si minimizza il tempo in cui bisogna creare qualcosa di nuovo. Risparmio generale di tempo.
			\item Si ha un vantaggio competitivo in termini di costo e tempi rispetto ai competitor che dovrebbero invece creare qualcosa da zero.
		\end{itemize}
		$\;\;$\textbf{Rischi:}
		\begin{itemize}
			\item Richiesto un investimento maggiore ed iniziale per mantenere archivi e librerie.
			\item Potrebbe venir violato il diritto di autore di chi ha sviluppato in passato.
			\item Si potrebbe "forzare" il riutilizzo dove non bisognerebbe farlo a causa di requisiti e bisogni differenti.
		\end{itemize}
	\end{info}
	\item \textbf{Business Process Diagramming}: Uso di strumenti grafici o anche non per descrivere processi e comprenderli meglio.
	\begin{info}[]
		\textbf{Punti di forza:}
		\begin{itemize}
			\item Approccio eccellente per i processi inter-funzionali;
			\item La comunicazione visuale è molto efficace.
			\item Si verifica efficacemente di cosa si sta parlando e di cosa non.
		\end{itemize}
		$\;\;$\textbf{Rischi:}
		\begin{itemize}
			\item L'implementazione dei miglioramenti dipende dall'apertura al cambiamento del committente. Inutile creare una soluzione che poi non verrà utilizzata;
			\item Richiede un buon supporto da parte degli esperti;
			\item Dispendioso dal punto di vista del tempo richiesto.
		\end{itemize}
	\end{info}
	\item \textbf{Prototyping}: Parte dell'incertezza si può risolvere fornendo prototipi da mostare al cliente.
	\begin{info}[]
		\textbf{Punti di forza:}
		\begin{itemize}
			\item Possono essere generate idee più innovative;
			\item Gli utenti riescono a chiarire meglio cosa vogliono;
			\item Gli utenti riescono meglio ad identificare requisiti dapprima ignorati;
			\item Il committente è centrale;
			\item Rapida verifica sulla fattibilità, utile anche per fare prove di performance;
			\item Stimola il processo mentale;
		\end{itemize}
		$\;\;$\textbf{Rischi:}
		\begin{itemize}
			\item Il cliente potrebbe voler l'implementazione del prototipo;
			\item Risulta difficile decidere quando fermarsi;
			\item Richiede skill;
			\item Assenza completa di documentazione;
		\end{itemize}
	\end{info}
	\item \textbf{Use Cases}: Anche se molto tecnico permette di capire bene come un processo funziona.
	\begin{info}[]
		\textbf{Punti di forza:}
		\begin{itemize}
			\item Lo stato del sistema è descritto prima che il committente interagisca col sistema;
			\item Lo stato del sistema è descritto utilizzando l'insieme completo degli scenari possibili;
			\item Il flusso degli eventi e delle eccezioni è rivelato facilmente;
		\end{itemize}
		$\;\;$\textbf{Rischi:}
		\begin{itemize}
			\item Le novità possono portare a incongruenze;
			\item Potrebbero essere ignorate delle informazioni;
			\item Le interazioni con il cliente possono essere lunghe;
			\item La formazioni per utilizzarla è costosa.
		\end{itemize}
	\end{info}
\end{itemize}
La Requirements Breakdown Structure è molto utile per trasformare i requisiti direttamente in attività. Inoltre è consistente con le indicazioni del PMBOK. SI tratta di un metodo piuttosto intuitivo e che consente anche al committente di capire in modo abbastanza semplice cosa si andrà a fare, mantenendo il rapporto attivo. Importante è non confondere RBS con WBS (Work Breakdown Structure) che invece viene utilizzata per definire le attività di progetto.\newline
\textbf{VANTAGGI DI RBS:}
\begin{itemize}
	\item Non richiede necessariamente un facilitatore esperto;
	\item Non richiede di imparare formalismi complessi;
	\item La raccolta dei requisiti è intuitiva;
	\item Il committente si trova in una comfort zone;
	\item Rappresenta con molta accuratezza quanto la soluzione è definita con chiarezza (per esempio colorando i requisiti con colori in base all'incertezza);
	\item Molto utile per scegliere il modello PMLC.
\end{itemize}
\subsubsection{Verifica degli attributi}
Come faccio a capire a che punto sono coi requisiti? Cosa mi serve chiarire? Le buone pratiche dicono che per ciascun requisito dobbiamo valutare determinati attributi. Ecco quali:
\begin{itemize}
	\item \textbf{Completezza}: I requisiti sono completi o ci siamo dimenticati di qualcosa?
	\item \textbf{Chiarezza}: I requisiti sono chiari? O sono ancora ambigui ed imprecisi?
	\item \textbf{Validità}: I requisiti riflettono l'intenzione del cliente? Il committente si aspetta del business value.
	\item \textbf{Misurabilità}: Esiste un criterio per valutare quanto un requisito soddisfa le richieste del cliente?
	\item \textbf{Verificabilità}: Può essere definito un criterio per verificare se un requisito fornisce la soluzione necessaria?
	\item \textbf{Manutenibilità}: L'implementazione è facile da manutenere?
	\item \textbf{Affidabilità}: I requisiti di affidabilità possono essere soddisfatti? (es. il server sarà sempre attivo?)
	\item \textbf{Look and Feel}: Sono stati soddisfatti i fattori relativi alla percezione richiesti dall'utente? (accessibilità, GUI, ergonomia etc.)
	\item \textbf{Fattibilità}: I requisiti possono essere implementati?
	\item \textbf{Precedenza}: C'è la possibilità di riuso da soluzioni con requisiti simili?
	\item \textbf{Scalabilità}: I requisiti sono articolati e complessi? Questo ce lo chiediamo per decomporre ulteriormente il requisito. La domanda da farsi è "possiamo scomporre ulteriormente?"
	\item \textbf{Stabilità}: Quanto questo requisito può cambiare? Quanto è stabile?
	\item \textbf{Performance}: Le prestazioni sono soddisfacenti?
	\item \textbf{Specifiche}: La documentazione prodotta è adeguata per supportare progettazione implementazione e test? Ho tutte le specifiche necessarie?
	\item \textbf{Sicurezza}: I requisiti di sicurezza sono pienamente dimostrati? Ci sono sia aspetti di cybersecurity, sia di sicurezza relativi al mondo reale (un forno che non si deve scaldare troppo etc.)
\end{itemize}
\subsubsection{La sfida nella gestione dei requisiti}
Il problema nella gestione dei requisiti sono molteplici. Non sono sempre ovvi, spesso bisogna capire bene cosa serve e cosa no. Spesso originano da diverse fonti e possono dunque essere in contrasto tra loro. I requisiti non sono semplici da descrivere a parole e possono essere ad un diverso livello di dettaglio, rendendo difficile la categorizzazione e le decomposizioni. Il numero dei requisiti può diventare ingestibile se non controllato, sia in termini di budget che di tempo. I requisiti non sono indipendenti tra loro, spesso esiste interrelazione tra requisiti, anche di diversi progetti. I requisiti infine sono soggetti al cambiamento nel tempo, sia a causa di leggi e normative che cambiano sia a causa di standard di qualità o desideri del cliente.
\subsubsection{Categorie di requisiti}
Esistono diversi tipi di requisiti:
\begin{itemize}
	\item \textbf{Funzionali}: specificano cosa il prodotto o servizio deve fare;
	\item \textbf{Non Funzionali}: mostrano le proprietà che dovrebbe avere il prodotto o servizio;
	\item \textbf{Globali}: sono i progetti di più alto livello inclusi nel prodotto o servizio. Si tratta di requisiti generali (es: l'architettura deve essere in cloud);
	\item \textbf{Di prodotto/Di progetto}: sono assimilabili ai vincoli di progettazione o di progetto.
\end{itemize}
\subsubsection{Alternative a RBS}
\paragraph{User Stories}
Devono catturare gli elementi essenziali di un requisito. I requisiti diventano tutti delle descrizioni sotto forma di frasi. Esistono diversi template un esempio è "\textit{As a \textbf{role}, I want to \textbf{action}; (so that \textbf{benefit})}". In questo caso con \textbf{role} definiamo chi è il protagonista del requisito, con \textbf{action} definiamo cosa ci si aspetta dal sistema e con \textbf{benefit} definiamo perché è importante (ha business value? quanto?).

\noindent Le user story non sono sufficienti, si fa fatica ad essere particolarmente espressivi, non si fornisce il contesto in modo chiaro e definito.
Una buona pratica è mantenere le user story abbastanza separate ed indipendenti tra loro, in modo da poter distribuirle meglio sugli sprint e bilanciare anche meglio i rischi. In questo caso il problema si sposta più sulle dipendenze.
Oltre alle user story possiamo definire anche:
\begin{itemize}
	\item \textbf{Epic}: Una user story particolarmente grande che necessita di più di uno sprint per essere completata. Può essere suddivisa in più user story;
	\item \textbf{Theme/Feature}: Molte user story correlate tra loro
\end{itemize}
Su questi elementi però c'è ancora abbastanza dibattito.
Quando si scrivono user story bisogna stare attenti a scriverle in modo che siano "interessanti". Importante è applicare dei criteri, valgono tutti i criteri che valgono anche in RBS. In più si possono utilizzare altri approcci, ad esempio l'improcio INVEST:
\begin{itemize}
	\item \textbf{Indipendent}: user story indipendenti sono più facili da pianificare;
	\item \textbf{Negotiable}: dialogando col cliente si riesce ad arrivare ad un accordo senza dover partire con specifiche molto stringenti;
	\item \textbf{Valuable}: Devono avere valore per il cliente (business value). Se una user story non ha business value è inutile;
	\item \textbf{Estimable}: facilmente stimabile;
	\item \textbf{Small}: abbastanza piccola per essere completata in un'iterazione. Nel caso in cui non lo siano abbastanza e non è possibile dividerle ulteriormente resta sempre possibile aumentare la durata dello sprint;
	\item \textbf{Testable}: già visto anche in RBS, se non è possibile effettuare test specifici per la singola user story probabilmente esistono dipendenze nascoste o che non sono state calcolate.
\end{itemize}
\paragraph{Role}
Si tratta dell'utente che interagisce con il sistema.

\noindent Importante da definire per l'assegnazione dei ruoli. L'identificazione dell'utente è importante per definire quali azioni sono possibili per ogni classe di utenti. Il team NON è mai il protagonista delle User Story. I protagonisti delle user story sono sempre gli user. Anche in questo caso esiste dibattito su questo punto. L'importante è chiarire come usare gli strumenti e non utilizzarli in maniera non ben definita.
\paragraph{Actions}
L'azione del sistema che effettua l'utente.

\noindent Di solito si definisce una sola azione per user story. Un suggerimento è utilizzare sempre la forma attiva rispetto alla passiva. L'importante è rimanere consistenti, si può utilizzare anche la forma passiva, ma in tal caso essa va mantenuta per ogni user story. Il "sistema" è sempre il protagonista implicito della user story.
\paragraph{Benefits}
Non necessariamente una user story corrisponde ad un solo beneficio. magari ci sono pochi benefici condivisi da più user story, o poche suer story che condividono più benefici comuni. Il beneficio non è necessariamente immediato per l'utente, può esserlo anche per altri stakeholders.

\paragraph{My little world}
Un esempio di dibattito su user story, feature, epic etc. è dato dal modello "my little world" di Marcel Britsch. Può essere utile sviluppare il concetto di \textit{Tema(Theme)} da immaginare come un obiettivo, una delle feature principali del sistema da sviluppare, suddivisibile in diverse epiche che sono a loro volta suddivisibili in user story. Possibile anche suddividere direttamente il tema direttamente in user story. Tutto questo ci porta ai requisiti.\newline
\centeredImage{document/img/mylittleworld.png}{My Little World - Prima versione del modello}{0.35}
\noindent Marcel però nel tempo ha cambiato idea sul modello. Per gestire realtà più complesse serve una struttura più complessa. Ora si parte dalle capabilities (funzionalità) del prodotto. Eventualmente è possibile definire dei temi (opzionalmente). Dato ciò che può fare il sistema si possono derivare delle features che eventualmente possono corrispondere a dei temi. La feature può essere implementata da una user story o ad un epic, che è a sua volta composta da user story. Il modello è fatto così perché è molto facile partire da cosa il sistema deve fare per poi arrivare a come suddividere su più livelli il lavoro.
\centeredImage{document/img/mylittleworld2.png}{My Little World - Versione aggiornata del modello}{0.75}

\subsubsection{User Stories vs. Condition of Satisfaction}
Un esempio trovato sul web: l'idea da cui partire è creare un sito per vendere birre. Sono state identificate quattro funzionalità:
\begin{itemize}
	\item Selezione di una birra automatico per il party;
	\item Scelta di nuove birre da assaggiare;
	\item Ordinare la birra preferita di nuovo;
	\item Raccomandare delle birre costose.
\end{itemize}
L'autore propone un pattern per descrivere le user story, il cui scopo è
\begin{itemize}
	\item Scegliere il nome della user story;
	\item Definire la user story;
	\item Gli associo una CoS.
\end{itemize}

\paragraph{Automatic Selection}
\textbf{Definizione}: As Jon (un manager che non ha tempo per selezionare le birre da portare al party), I want to get beers selected by the system, so that I can impress my friends by variety of rare brands.\newline
\textbf{CoS}: Jon può impressionare i suoi amici, facendo un figurone selezionando diverse birre da diverse parti del mondo.\newline

\paragraph{New beers to taste}
\textbf{Definizione}: As Jon, I want to see beer catalog so I can choose the some new one to taste.\newline
\textbf{CoS}: Jon può vedere i sapori direttamente dal catalogo senza dovere scendere nei dettagli della birra.\newline

\paragraph{Favorite beer order}
\begin{info}
	Bisognerebbe definire cosa si intende con Favorite, dato che una birra potrebbe essere chiamata in questo modo se rispetta determinati canoni (comprata il maggior numero di volte da sempre, nelle ultime due settimane, etc.)
\end{info}
\textbf{Definizione}: As a returning customer I want to see my favorite beers, so I can order them again.\newline
\textbf{CoS}: vuoto (è un parametro opzionale, ma la mancanza di una CoS potrebbe indicare la mancanza di business value, secondo l'autore questo parametro di può omettere, ma è questione di opinioni).\newline

\paragraph{Recommend expensive beers}
\textbf{Definizione}: As a store owner, I want the system to recommend expensive beers, so we can increase our profit.\newline
\textbf{CoS}: Non pressare troppo l'utente. Bisogna stare attenti a non essere troppo invadenti, proporre qualche birra costosa ma non troppo spesso. \newline

\begin{warn}
	\textbf{Quando è meglio formulare le CoS? Prima o dopo la definizione di user stories?}
	Come al solito è questione di opinioni, dipende anche dalla user story. Non c'è una vera e propria regola, a volte è più facile definire prima la definizione per poi tirare fuori la CoS, mentre a volte è proprio la CoS che ci aiuta nella definizione della user story.
\end{warn}

\subsection{Business process : l’uso dei diagrammi}
\subsubsection{Supporto alla raccolta dei Requisiti}
A volte il committente non riesce a immaginare bene la soluzione e i senior non trovano valore economico nel progetto. In questo caso si può supportare il processo di raccolta dei requisiti con diagrammi di processi aziendali, workflow, prototipizzazione, verifica dell'impatto della soluzione, SWOT analysis.
\subsubsection{Business Process}
Un processo aziendale è un insieme di attività con dettaglio precisate che prendono dell'input (può essere per esempio un ordine o un materiale, può essere di varia natura, concreto o astratto) da più sorgenti e producono un cambiamento di stato che fornisce business value.
\centeredImage{document/img/businessprocess.png}{Business Process}{0.5}
\subsubsection{Business Process Diagram}
\centeredImage{document/img/businessprocessdiagram.png}{Simboli del Business Process Diagram}{0.4}
\noindent Esistono diversi formati, nel seguente si legge dall'altro verso il basso e da sinistra verso destra.
\centeredImage{document/img/topdownleftright.png}{Top-Down, Left-Right Format}{0.5}
\noindent In questo formato esiste una corsia per ogni elemento che viene coinvolto nel processo.
\centeredImage{document/img/swimlane.png}{Swim-Lane Format}{0.5}
\noindent In questo formato invece si cerca di chiarire meglio il contesto.
\centeredImage{document/img/contextdiagrammingprocess.png}{Context Diagramming Process}{0.5}
\noindent Diverse rappresentazioni possono donarci diverse informazioni, il tipo di rappresentazione dipende fortemente dal tipo di processo. In azienda
\subsection{Business Process Management}
I processi di business devono essere gestiti e devono essere progettati. Lavorare sui processi aziendali in maniera iterativa è una modalità. Una volta costruito il modello è possibile simularlo per vedere se i parametri impostati rispettano le aspettative. Una volta dopo aver simulato e appurato il funzionamento si passa all'implementazione, all'esecuzione e al monitoring. L'infrastruttura tecnologica può cambiare molto rapidamente. Approcci utilizzabili sono AGILE e DevOps.

\subsubsection{Workflow}
Il \textbf{workflow} è l'automazione di un processo di business totale o parziale, durante la quale documenti, dati etc. sono scambiati tra entità coinvolte per svolgere con il supporto di un sistema software le attività previste.\newline
Il primo passo per automatizzare è fornire la definizione dei processi di business. A volte è necessario modificarli per poterli automatizzare.

\subsubsection{Proof of Concept, Prototipo, MVP}
Si potrebbe essere incerti su ciò che può essere il risultato. Abbiamo già detto che la prototipizzazione può esser un possibile modo per chiarire le idee. Non esiste però un solo modo per prototipizzare:
\begin{itemize}
	\item \textbf{Proof of Concept (PoC)}: utilizzato quando è necessario verificare che una certa idea è fattibile dal punto di vista tecnico. Per esempio quando è necessario utilizzare una nuova tecnologia e si è incerti sul successo. Sarebbe necessario rimandare il più possibile gli aspetti tecnici, ma in questo caso è meglio capire fin da subito quanto gli aspetti tecnici sono rischiosi. La tecnologia in questo caso è una CoS molto importante e ha molto peso. In questo caso non si crea l'intero sistema, ma si cerca di minimizzare il lavoro, cercando di capire se i punti critici risultano fattibili, realizzando ed applicando per esempio l'algoritmo su cui si hanno incertezze su dati che non sono necessariamente appartenenti al dominio di interesse. Se analizzando le performance si ottengono buoni risultati allora attenuiamo il rischio. Punti chiave sono la minimizzazione del lavoro, ma è necessario simulare il sistema in modo che sia il più vicino possibile a quello reale, almeno nei punti critici.
	\item \textbf{Prototipo}: Significa creare il prodotto desiderato in una sua rappresentazione per mostrare l'aspetto e le modalità d'uso all'utente per poter identificare problemi in modo da poter apportare eventuali correzioni e miglioramenti il prima possibile.
	\begin{warn}
		Sia PoC che il prototipo hanno in output un sistema o un prodotto funzionante, che tuttavia non può essere utilizzato come punto di partenza per il progetto vero e proprio. I risultati non possono essere utilizzati in alcun modo e vanno gettati non appena il loro scopo viene raggiunto.
	\end{warn}
	\item \textbf{Minimum Viable Product (MVP)}: C'è più lavoro da fare per mostrare se il prodotto raggiunge le aspettative del cliente. Si tratta di una prima versione del software in cui vengono implementate solo le funzioni principali. In questo caso il prodotto non viene gettato come accade negli altri due casi ma viene utilizzato per continuare a sviluppare il prodotto finale. In questo caso il codice deve essere di qualità sin dalla prima release.
\end{itemize}
\subsubsection{SWOT Analysis}
Una tecnica di analisi e di pianificazione strategica impiegata per poter identificare quattro componenti, due dei quali sono fattori interni mentre gli altri sono esterni:
\begin{itemize}
	\item \textbf{Strenghts (punti di forza)} (interna): caratteristiche del progetto che possono darci un vantaggio rispetto alla concorrenza (es: saper utilizzare molto bene una tecnologia, avere una clientela fissa, staff skilled, etc.).
	\item \textbf{Weaknesses (debolezze)} (interna): caratteristiche del progetto che possono darci svantaggio rispetto alla concorrenza (es: mancanza di risorse, mancanza di conoscenza su tecnologie, disorganizzazione, etc.).
	\item \textbf{Opportunità (opportunities)} (esterna): elementi relativi al mondo esterno che possono essere sfruttate per portare vantaggio al progetto (es: aree del mercato non coperte etc.).
	\item \textbf{Threats (minacce)} (estrena): elementi relativi al mondo esterno che possono causare problemi nel progetto (es: nuovi competitors, cattive recensioni, cambiamenti di regole, etc.).
\end{itemize}
Esistono varie rappresentazioni per la SWOT analysis, a quadranti, ad elenchi etc.
\subsection{Modello di  Business}
Descrive le modalità con cui l'organizzazione intende creare distribuire e raccogliere valore. Il modello di Business deve interessarci, perché dobbiamo capire come distribuire il budget. Quando dobbiamo vendere un prodotto non pensiamo solo al quanto incasseremo dalla vendita ma anche al come dovremmo vendere il prodotto. Un modello di business di esempio: il cliente paga per un software che sarà suo per sempre. Un modello di business differente può essere ad esempio una licenza che scade mensilmente. Anche le modalità con cui va fornita assistenza dipende dal modello di business. Tutte queste scelte hanno un impatto sul quantitativo di denaro a disposizione dell'azienda.

\noindent Il modello di Business non va visto come un singolo concetto macro, anche una revisione interna di un processo aziendale in cui nessun cliente esterno è interpellato influisce sul modello di business.

\noindent Un approccio molto interessante e facile da usare è il modello Canvas. Non vedremo come funziona a livello approfondito, ci basterà l'aspetto visuale:
\centeredImage{document/img/businessmodel.png}{Modello di Business}{0.5}
\noindent Parte definendo il valore che noi offriamo ad un cliente. Il valore va definito come business value. Il cliente è necessario raggiungerlo in qualche modo, per farlo esistono diversi canali possibili. Poi è necessario instaurare un rapporto con il cliente. Tutto quello che riguarda il rapporto col cliente incide sul fatturato. Anche la scelta dei cliente impatta. Per produrre valore ho bisogno di attività chiave, di persone, partner chiave.
\subsection{Scegliere il PMLC Model}
Il grado di completezza della RBS è il fattore più importante per decidere il modello da utilizzare. Più è alta l'incertezza e più dovremmo stare lontani da un approccio classico ed iterativo.\newline
Solitamente i requisiti di alto livello sono quelli che forniscono in modo diretto la maggior parte del business value. Facciamo ancora una volta un recap:
\begin{itemize}
	\item \textbf{Lineare}: La soluzione ed i requisiti sono definiti molto chiaramente. Non sono previsti molti cambiamenti di scope. Il progetto è di routine, molto ripetitivo. Possono essere utilizzati template.
	\item \textbf{Incrementale}: Le condizioni sono le stesse dell'approccio lineare ma il cliente desidera procedere in maniera più incrementale. Questo accade perché possono essere più probabili le richieste di cambio dello scope.
	\item \textbf{Iterativo}: Si pensa che i requisiti possano cambiare facilmente. Alcuni requisiti verranno scoperti in corso d'opera. Alcuni aspetti della soluzione non sono neanche stati identificati quando si parte.
	\item \textbf{Adattivo}: La soluzione ed i requisiti si conoscono solo parzialmente. Alcune funzionalità non sono ancora state identificate. I cambiamenti di scope sono praticamente sicuri.  Il progetto è orientato allo sviluppo di un nuovo prodotto o al miglioramento di un processo già esistente. Mancanza di tempo e deadline stringenti che non permettono un re-plan.
	\item \textbf{Extreme}: I goal e le soluzioni non sono affatto chiare. Il progetto è di tipo R \& D.
\end{itemize}
\subsection{Scrivere il Project Overview Statement (POS)}
Lo step finale è costruire un documento, una descrizione sintetica del progetto (Wysocki parla di una pagina "singola"). Il grado di sintesi deve essere estremo, ma deve rappresentare:
\begin{itemize}
	\item Una \textbf{dichiarazione generale} che descrive in cosa consiste il progetto. Quali sono gli obiettivi e i goal?;
	\item Un \textbf{riferimento} per il team di pianificazione. L'importante è non perdere l'integrità concettuale. Deve essere chiaro in modo da capire chiaramente di cosa stiamo parlando;
	\item Un \textbf{aiuto per le decisioni} riguardanti il progetto. A volte alcune idee vengono definite chiaramente sin da subito e non dovrebbero essere disperse;
	\item Il documento utilizzato per ottenere l'approvazione del progetto e il nulla osta a procedere con la pianificazione.
\end{itemize}
Un documento alternativo al POS si chiama Project Charter.
\subsubsection{Contenuto del Project Overview Statement}
Indipendentemente da come verrà organizzato il documento, questi cinque punti ci devono essere:
\begin{itemize}
	\item \textbf{Problem/Opportunity}: Un progetto può aiutare a risolvere un problema, a prendere un'opportunità o ad un mix dei due.
	\item \textbf{Project Goal}: Potremmo avere un problema o un opportunità ma possiamo catturarlo in mille modi diversi. Ci sono tanti possibili goals dato il \textbf{Problem/Opportunity}. Importante è che il Goal sia correlato. Il goal non può non essere allineato con il problema o l'opportunità. Di solito si tratta di una parte piuttosto sintetica. Bastano uno o due statement.
	\item \textbf{Project Objectives}: Definiscono meglio quali saranno gli obiettivi da perseguire dato il \textbf{Goal}. Anche qua obiettivi e goal devono essere allineati. Bisogna avere dei riferimenti per capire cosa è classificabile come obiettivo. Deve essere chiaro cosa è incluso e cosa no. In questa parte va definito anche quali sono i deliverables (anche la documentazione). Anche in questo punto possono essere utili dei criteri, un possibile modo è utilizzare i criteri S.M.A.R.T:
	\begin{itemize}
		\item \textbf{Specific}: essere specifici nel definire l'obiettivo.
		\item \textbf{Measurable}: i progressi devono poter essere misurati in maniera quantificabile.
		\item \textbf{Assignable}: definire chi deve conseguire il completamento di un obiettivo.
		\item \textbf{Realistic}: deve rispondere a qualcosa che è perseguibile.
		\item \textbf{Time-Related}: deve essere stimabile l'attività necessaria e capire quanto tempo servirà, in modo da poter stabilire deadline e capire se è fattibile.
	\end{itemize}
	Questi sono solo degli esempi, anche in questo caso le keyword possono essere interpretate in modo differente.
	\item \textbf{Success Criteria}: Criteri semplici spesso allineati con gli obiettivi ma non necessariamente. Ci aiuta a rispondere alla domanda "In base a quale criterio dirò che il risultato è soddisfacente?". I criteri devono essere necessariamente quantitativi perché misurabili. Devono essere legati al concetto di business value. Spesso possono essere espressi in termini di:
	\begin{itemize}
		\item \textbf{Increased Revenue (IR)}: aumento delle entrate;
		\item \textbf{Avoided Costs (AC)}: costi evitati;
		\item \textbf{Improved Services (IS)}: miglioramento dei servizi;
	\end{itemize}
	\item \textbf{Assumptions/Risks/Obstacles}: Potrebbero essere sezioni indipendenti. Le assunzioni sono cose che si definiscono e si da per scontato che ci saranno (es: il team deve lavorare full time). I rischi e gli ostacoli sono collegati. Gli aspetti più ricorrenti:
	\begin{itemize}
		\item \textbf{tecnologici}: nuove tecnologie o tecnologie deprecate;
		\item \textbf{ambienti}: cambi di personale;
		\item \textbf{interpersonali}: relazioni tra le persone;
		\item \textbf{culturali}: l'ambito del progetto è conveniente?;
		\item \textbf{relazioni causali}: la soluzione risolverà il problema?
	\end{itemize}
\end{itemize}
In questo documento mancano tutti i requisiti. Non sono importanti, questo è un documento di sintesi, i requisiti stanno altrove. Il senior management che legge questo documento non è interessato ai dettagli.

\subsubsection{Esempi di Goal Statement}
Notare come ogni singola parte delle frasi è importante.
\begin{itemize}
	\item \textbf{Apollo}: "Far sbarcare un uomo sulla luna entro la fine della decade e farlo ritornare in modo sicuro sulla terra.";
	\item "Creare un nuovo sito web per la nostra compagnia per aumentare le vendite online del 20\% entro il prossimo anno.";
	\begin{info}
		In questo caso non si capisce neanche il contesto, se è un problema, una opportunità etc. ma l'obiettivo è chiaro. Notare come non si specifichi che il sito sia un eCommerce, potrebbe anche non dare la possibilità di acquisti online e potrebbe anche solo sponsorizzare. Questi dettagli non stanno qui.
	\end{info}
	\item "Realizzare un software di ERP per la nostra compagnia entro un anno per aumentare la soddisfazione dei dipendenti e le performance dei processi di almeno 30\%";
	\begin{info}
		Potrei doverlo cambiare perché non sono contento del sistema attuale. In questo caso il termine performance è molto vago. Da qualche altra parte deve essere spiegato e definito meglio il concetto di performance, che potrebbe essere definito in modo differente da figure differenti.
	\end{info}
	\item "Sviluppare un nuovo strumento di ricerca per poter utilizzare della conoscenza esistente a disposizione, in un anno posso accumulare fino a un milione di query e voglio che l'utente sia soddisfatto di almeno uno dei top tre risultati offerti dalla query, per il 90\% delle query";
	\begin{info}
		Da questo goal si evincono chiaramente anche alcuni dettagli del sistema, che sono stati descritti per constatare quale fosse l'obiettivo da raggiungere. In questo caso essendo i vincoli molto stringenti serve un sistema di monitoraggio più potente di un semplice questionario.
	\end{info}
	\item "Aggiornare il servizio del sistema clienti entro il 31 ottobre per raggiungere una media di tempo di attesa di non più di due minuti."
		\begin{info}
		Si evince che il sistema esiste già. Misurare in questo caso è più facile. Sistemi di questo tipo non sono semplici da calibrare e serve chiarire bene in che modo affrontare il problema in caso di picchi di chiamate. Come si può affrontare il problema? Ad esempio avvisando direttamente il cliente con un messaggio preconfigurato in cui si avverte che ci sono problemi.
	\end{info}
\end{itemize}
\noindent Non è necessario che stia davvero tutto in una paginetta, in progetti più complessi ne servono alcune. Gli elementi non inclusi nel POS perché troppo specifici e dettagiati devono essere inclusi nel Project Definition Statements (PDS). La lista di elementi che può essere inserita in questo documento è molto estesa. Ogni elemento raccolto in più è un buon input per le fasi successive. L'importante resta sempre il non perdere l'integrità concettuale.
\subsubsection{Allegati POS}
Si possono includere alcuni allegati:
\begin{itemize}
	\item \textbf{Analisi dei rischi}
	\item \textbf{Analisi finanziaria}: che include studi di fattibilità, analisi dei costi e benefici. analisi di breakeven, ritorno degli investimenti...
\end{itemize}

\subsubsection{Alternative - Project Charter}
PMBOK propone di utilizzare il Project Charter. Si tratta di un altro tipo per riassumere il POS ma ha una struttura leggermente diversa. Il documento autorizza formalmente l'esistenza del progetto. Una volta approvato il documento si può procedere col planning.
Il Project Charter stabilisce cosa dovrà essere il progetto e tra le altre cose stabilisce la partnership tra chi svilupperà la soluzione e chi la richiede.
Questo documento NON è un contratto, il contratto verrà stipulato in caso di progetto esterno ma si tratta di un documento diverso. Col Project Charter specifichiamo come l'organizzazione raggiungerà la soluzione. Il documento una volta che viene approvato definisce la soluzione, ogni volta che si cambia idea sulla soluzione bisogna tornare al processo di scoping, e quindi a modificare anche questo documento.
Gli input potrebbero essere dei documenti che insieme compongono il Project Charter ed includono:
\begin{itemize}
	\item \textbf{Statement of work (SOW)}: è la descrizione del risultato che vogliamo ottenere. Non è ben specificato in che modo va scritto, l'importante è che le idee siano chiari per ogni parte.
	Per progetti interni si fornisce il SOW in funzione dei bisogni dell'azienda, mentre per progetti esterni può essere una richiesta documentata dal cliente stesso (es: RFP, RFI, un contratto, un bando). Un SOW può essere composto dai seguenti elementi:
	\begin{itemize}
		\item \textbf{Business need}: i bisogni aziendali (o ci vengono forniti o serve una fase di analisi).
		\item \textbf{Product scope description}: La keyword Product potrebbe essere anche un servizio o qualsiasi altra cosa. Bisogna determinare le funzionalità e le caratteristiche del prodotto.
		\item \textbf{Strategic plan}: un piano strategico relativo al progetto, ossia sul medio e lungo periodo e non solo sul breve (deve essere allineato al piano strategico aziendale). Anche chi non è ad un livello gerarchico alto deve capire bene la visione  aziendale in modo da poterla assecondare nel migliore dei modi.
	\end{itemize}
	\item \textbf{Business case}: fornisce tutte le informazioni necessarie per determinare se il progetto vale o meno l'investimento richiesto. Ci si può rifiutare di fare un progetto se non porta abbastanza benefici, puntando su altri progetti più proficui. Può essere creato utilizzando i seguenti elementi:
	\begin{itemize}
		\item \textbf{bisogni di mercato}: il mercato in generale richiede determinate tecnologie, sostanzialmente a volte va "seguita la moda";
		\item \textbf{bisogni dell'organizzazione}: a volte il progetto può portare benefici all'azienda;
		\item \textbf{richiesta del cliente}:  bisogna verificare se la richiesta del cliente può essere accolta. A volte non è conveniente;
		\item \textbf{bisogni tecnologici}: l'azienda non può utilizzare una vasta gamma di tecnologie se poi le usa male, meglio poche ma buone. Per questo non si accettano tutti i progetti, potremmo andare in perdita per i troppi costi dovuti alla formazione;
		\item \textbf{requisiti legali}: i vincoli legali che vengono imposti sul contratto e l'ambiente esterno vincolano il progetto;
		\item \textbf{impatto ecologico}: vincoli su quali mezzi utilizzare per limitare l'impatto ecologico; Non è solo una questione ideologica ma anche legale;
		\item \textbf{bisogni sociali}: bisogni delle persone in generale.
	\end{itemize}
	\item \textbf{Agreements}: Molto spesso quando partiamo a definire cosa dobbiamo inserire in un progetto ci sono già degli accordi tra le parti. Potrebbero essere contratti, collaborazioni in alcuni ambiti, degli accordi sul livello di servizio (Service Level Agreeents (SLA)), memorandum d'intesa (Memorandum of Understanding (MOU)), lettere di intenti e accordi, accordi verbali (informali).
	\item \textbf{Enterprise enviromental factors}: sono i fattori ambientali aziendali che possono influenzare il processo di sviluppo descritti nel project Charter, sono inclusi standard di vario tipo (governativi, di settore), regolamenti, cultura e struttura organizzativa, condizioni di mercato.
	\item \textbf{Organizational process assets}: Sono gli asset in termini di processi organizzativi, includono le definizione dei processi, le linee guida, l'utilizzo di templates, informazioni storiche, utilizzo di esperienza passata etc.
\end{itemize}
\subsubsection{Elementi per un buon Project Charter}
Un buon Project Charter deve contenere in particolare alcuni elementi. Non si tratta di un elenco ordinato, l'importante è la presenza di tutti questi punti:
\begin{itemize}
	\item \textbf{obiettivi di progetto misurabili e relativi criteri di successo}: a fine progetto, in fase di pianificazione ed esecuzione ci servono per poi essere sicuri che i risultati vengono raggiunti;
	\item \textbf{requisiti di alto livello}: potrebbero essere gli obiettivi o un dettaglio ulteriore da aggiungere. Ad esempio possiamo mettere sia il primo che il secondo livello di un RBS;
	\item \textbf{assunzioni e vincoli}: incidono pesantemente sui vincoli e su come verranno implementati;
	\item \textbf{descrizione di alto livello del progetto e dei suoi confini (boundaries)}: un buon riferimento, soprattutto durante la fase di pianificazione;
	\item \textbf{rischi di alto livello}: i rischi più rilevanti per un senior manager;
	\item \textbf{riepilogo delle milestone di progetto}: la milestone è un punto in cui bisogna arrivare in cui un certo numero di requisiti sono soddisfatti. Questo aspetto è un po' prematuro, ma ha senso farlo qua perché ci sono anche delle "milestone concettuali" per capire cosa è importante cosa no. Ad esempio si possono raggruppare alcuni requisiti perché ci sono forti dipendenze per cui nessuno può essere utilizzato senza che anche altri siano presenti;
	\item \textbf{riepilogo del budget}: e per cosa utilizzarlo;
	\item \textbf{elenco degli stakeholder}: non c'era nel PoS, ma è una buona idea segnare ogni stakeholder.
	\item \textbf{requisiti per l'approvazione del progetto}: diverso dai criteri di successo. Potrebbe mappare i criteri di accettazione, ma può essere utilizzato per capire come fare i test.
	\item \textbf{project manager assegnato, responsabilità e livello di autorità}: spesso si ha già una pre-assegnazione. Non è detto che le risorse utilizzate durante la fase di planning etc. siano le stesse che poi lavoreranno al progetto.
	\item \textbf{Nome e autorità dello sponsor e di tutte le "persone" che autorizzano il project charter}: A volte non si scrive chi è lo sponsor perché spesso dalle riunioni si sa, ma in una versione interna è ottimo specificarlo per farlo sapere anche a chi magari non ha partecipato.
\end{itemize}
\subsection{Approval Process}
Il senior manager potrebbe farsi alcune domande per valutare il progetto, sarebbe ottimo riuscire a rispondere in anticipo (non è un elenco esaustivo):
\begin{itemize}
	\item \textbf{Quanto è importante il problema o l'opportunità per l'organizzazione?}
	\item \textbf{Quanto il progetto è legato ai Critical Success Factors aziendali?}: sarebbero gli interessi dell'azienda, i desideri, i piani strategici dell'azienda ci dicono cosa è importante e dobbiamo capire se il progetto è legato ai nostri valori e fattori.
	\item \textbf{Il goal statement è legato direttamente al problema/opportunità?}: il goal deve essere allineato con il problema.
	\item \textbf{Gli obiettivi sono una chiara rappresentazione del goal statement?}: Stessa cosa della domanda precedente, ma la domanda è posta in termini degli obiettivi.
	\item \textbf{C'è sufficiente business value, come misurato dai criteri di successo per giustificare i costi del progetto?}: A volte si può andare in perdita per un vantaggio che si otterrà in futuro (investimento).
	\item \textbf{La relazione tra gli obiettivi del progetto e i criteri di successo è stabilita chiaramente?}: gli obiettivi devono quadrare con un qualche cosa di misurabile che mi dice che sono riuscito nei miei intenti.
	\item \textbf{I rischi sono troppo alti e il business value troppo basso?}: mi prendo troppi rischi rispetto al valore atteso?
	\item \textbf{Può il senior management mitigare i rischi che sono stati identificati?}: Se il progetto è interessante ma ci sono troppi rischi che potrebbe essere mitigati con delle richieste (questo progetto si può fare, ma ci serve almeno il 20\% di budget in più)
\end{itemize}
