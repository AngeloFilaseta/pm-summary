\section{Closing Process group}
\subsection{Tools, templates e processi per chiudere il progetto}
Per ottenere che la chiusura del progetto sia un processo ben definito e ordinato sono molto utili i seguenti strumenti, template e processi:
\begin{itemize}
	\item \textbf{Strategie di implementazione};
	\item \textbf{Procedure di collaudo};
	\item \textbf{Documentazione di progetto}: Questa operazione è più una fase di check in cui si aggiungono informazioni mancanti, non si scrive tutta la documentazione in questa fase;
	\item \textbf{Audit post-implementazione}: Una retrospettiva generale del progetto;
	\item \textbf{Rapporto finale di progetto}.
\end{itemize}

\subsection{Chiudere il progetto}
Dopo aver ottenuto l’approvazione del committente, a fronte di un eventuale collaudo, la chiusura di un progetto è un processo di routine che prevede i seguenti passi:
\begin{itemize}
	\item Ottenere l’accettazione formale del deliverable;
	\item Assicurarsi che tutti i deliverable siano stati installati;
	\item Assicurarsi che tutta la documentazione necessaria sia disponibile;
	\item Ottenere la firma del committente sul report finale;
	\item Condurre l’audit post-implementazione;
	\item Celebrare la fine (con successo… si spera) del progetto.
\end{itemize}

\subsection{Ottenere l’accettazione del committente}
Una procedura di accettazione ben definita e chiara evita che si possano verificare dei fraintendimenti e che si scoprano eventuali problemi alla chiusura del progetto. I criteri di accettazione devono essere chiaramente definiti con la collaborazione del committente durante la fase di planning. In fase di esecuzione il team si deve accertare che il deliverable prodotto possa soddisfare tali criteri. Un “collaudo” dovrebbe certificare il rispetto dei criteri stabiliti. Se i deliverable soddisfano i criteri di accettazione non dovrebbe accadere che il committente non sia contento.

\subsection{Installare la soluzione}
Una volta accettato il deliverable si può procedere all’installazione per poter effettuare il “go live” (mandare in produzione). Possono essere seguiti i seguenti approcci:
\begin{itemize}
	\item \textbf{Phased Approach}: decompone il deliverable in tante parti (fasi) che saranno sviluppate e consegnate in sequenza. Questo approccio non è molto adatto per "rimpiazzare" un sistema;
	\item \textbf{By Business Unit Approach}: il deliverable è installato in una business unit alla volta. Un approccio molto simile può prevedere l’installazione progressiva del deliverable per aree geografiche;
	\item \textbf{Cut-Over Approach}: prevede un’immediata sostituzione del vecchio deliverable con il nuovo. Il giorno che si decide di effettuare il cambio si spegne il vecchio sistema e si accende il nuovo. Questo approccio richiede necessariamente che l’ambiente di test sia identico all’ambiente di produzione, inoltre è molto rischioso;
	\item \textbf{Parallel Approach}: il nuovo deliverable è installato mentre la “vecchia” soluzione è ancora operativa. Dopodiché, le due soluzioni saranno in produzione simultaneamente fino a quanto non sarà verificato il corretto funzionamento di quella nuova. Questa infrastruttura è piuttosto costosa.
\end{itemize}
\subsection{Documentare il progetto}
Documentare il progetto è una delle attività più difficili e “faticose” da completare. Comunque ci sono valide ragioni per documentare in modo adeguato un progetto, ad esempio:
\begin{itemize}
	\item La documentazione è un riferimento fondamentale per riuscire ad apportare modifiche future a un deliverable o permettere il riuso di alcune sue parti. Tramite la documentazione possiamo risparmiare molto lavoro, magari perché viene riutilizzata una parte di un altro progetto;
	\item È una sorgente di dati storici per poter effettuare le stime dei tempi, dei costi e dei rischi per le attività dei progetti futuri;
	\item È una valida risorsa per il training di nuovi project manager;
	\item È un valido supporto anche per il training e la crescita professionale del team di progetto;
	\item Fornisce un input per la valutazione delle performance dei membri del team da parte del functional manager.
\end{itemize}
\subsubsection{Project Notebook}
Segue una lista di tutta la documentazione di progetto in chiusura:
\begin{itemize}
	\item \textbf{POS};
	\item \textbf{RBS comprendendo tutte le sue revisioni};
	\item \textbf{Proposta contenente i requisiti di progetto};
	\item \textbf{Schedule del progetto: originale (fase di planning) e correzioni};
	\item \textbf{Appunti/verbali di tutti i project team meeting};
	\item \textbf{Copia di tutti gli status reports};
	\item \textbf{Documentazione relativa al design};
	\item \textbf{Eventuali prototipi o sample realizzati};
	\item \textbf{Copia di tutti gli avvisi di modifiche};
	\item \textbf{Copia di tutte le comunicazioni scritte};
	\item \textbf{Report delle questioni rimaste in sospeso};
	\item \textbf{Rapporto finale};
	\item \textbf{Documenti di accettazione del committente};
	\item \textbf{Rapporto dell’audit post-implementazione}.
\end{itemize}
\subsection{Condurre l’audit post-implementazione}
Le domande da porsi in questa fase sono le seguenti:
\begin{itemize}
	\item \textbf{Gli obiettivi del progetto sono stati raggiunti?};
	\begin{itemize}
		\item \textbf{Il deliverable fa quello che aveva previsto il team?}: La risposta potrebbe anche essere no, ma magari perché si è trovato un accordo con il cliente;
		\item \textbf{Il deliverable fa quello che si aspettava il committente?}
	\end{itemize}
	\item \textbf{Il progetto è stato completato rispettando i limiti di tempo, budget e rispettando le specifiche?}: non bastano tempo e denaro, serve anche aver rispettato le specifiche. In caso negativo è necessario capire perché, in modo da fare meglio in futuro.
	\item \textbf{Il committente è soddisfatto del risultato del progetto?}: il cliente può non essere contento anche perché tutti i requisiti sembrano rispettati. Anche in questo caso bisogna capire perché. Potrebbe essere un problema che riguarda la sua user experience.
	\item \textbf{Il business value previsto si è concretizzato?}
	\begin{itemize}
		\item \textbf{I criteri di successo sono stati rispettati?}: a volte no, magari però abbiamo fatto bene a non rispettarli.
	\end{itemize}
	\item \textbf{Che lezione è stata imparata relativamente alla metodologia di gestione del progetto scelta?}: è stata una buona idea insistere con il metodo tradizionale?
	\item \textbf{Come ha seguito la metodologia il team?}: utile per capire se è necessario qualche cambiamento, ad esempio nelle durate dello sprint etc.
\end{itemize}
A volte (spesso) la fase di audit post implementazione non viene svolta per i seguenti motivi:
\begin{itemize}
	\item I manager non vogliono sapere cosa è avvenuto durante il progetto;
	\item I manager non vogliono pagare il costo dell’audit;
	\item L’audit non è percepita come un’attività ad alta priorità;
	\item C’è molto altro lavoro da fare (su altri progetti).
\end{itemize}
\subsection{Scrivere il rapporto finale}
Il report finale è atto a dare le informazioni più importanti riguardanti il progetto. Si tratta principalmente di un sommario rivolto al project manager che ha una visione non dettagliata di tutto quello che è successo. Il documento ha la seguente struttura:
\begin{itemize}
	\item \textbf{Executive Summary};
	\item \textbf{Livello di successo e performance complessive del progetto}: se si hanno dei KPI piuttosto standardizzati è possibili fornire informazioni facilmente leggibili e significative in un colpo solo;
	\item \textbf{Organizzazione e amministrazione del progetto}: quale approccio e stato utilizzato e come è stato personalizzato per il progetto;
	\item \textbf{Tecniche impiegate per raggiungere il risultato}: specificità che riguardano i risultati del singolo progetto, anche dal punto di vista tecnico;
	\item \textbf{Pregi e difetti dell’approccio utilizzato}: sia dal punto di vista architetturale che di gestione del progetto;
	\item \textbf{Raccomandazioni};
	\item \textbf{Appendici}: si tratta sostanzialmente di riferimenti ad altri documenti;
	\begin{itemize}
		\item \textbf{POS};
		\item \textbf{WBS};
		\item \textbf{Schedulazione delle risorse};
		\item \textbf{Richieste di cambiamenti};
		\item \textbf{Deliverable finali};
		\item \textbf{Altro}.
	\end{itemize}
\end{itemize}
