\section{Definizione dei Processi}
\subsection{Definizione dei 5 Gruppi di Processi (Process Groups)}
\subsubsection{Scoping/Initiating Process Group}
\begin{info}
	Sviluppa e ottiene l'approvazione di una dichiarazione generale riguardante obiettivi e business value del progetto. In questa fase si sceglie se il "cosa" realizzare va bene a tutti.
\end{info}
I processi consentono di:
\begin{itemize}
	\item Documentare i veri bisogni del cliente. Lo scopo è utilizzare la documentazione come strumento di comunicazione. Inoltre è utile per la creazione del contratto;
	\item Negoziare con il cliente la modalità con cui dovranno essere soddisfatti i suoi bisogni;
	\item Descrivere il progetto in modo sintetico;
	\item Ottenere l'approvazione del \textit{Senior Management}.
\end{itemize}
\subsubsection{Planning Process Group}
\begin{info}
	Identifica le attività che devono essere svolte per implementare i requisiti e completare il progetto, stima il tempo, i costi e le risorse necessarie e ottiene l'approvazione del piano. Anche il planning ha bisogno di approvazione. In questo Process Group rientra anche l'analisi dei rischi. In questa fase si sceglie se il "come" realizzare va bene a tutti.
\end{info}
I processi consentono di:
\begin{itemize}
	\item Definire tutte le attività del progetto;
	\item Stimare tempi, costi e risorse;
	\item Sequenziare le attività e costruire la "schedula" iniziale del progetto (il calendario delle attività). Sarà sempre possibile eventualmente analizzare e modificare la schedula in questa fase;
	\item Scrivere il piano di gestione del rischio;
	\item Documentare il piano;
	\item Ottenere l'approvazione del \textit{Senior Management} per eseguire il progetto.
\end{itemize}
\subsubsection{Launching/Executing Process Group}
\begin{info}
	Si seleziona il personale da coinvolgere nel team di progetto, si stabiliscono le regole operative del team e si aiuta il team a lavorare insieme.
\end{info}
I processi consentono di:
\begin{itemize}
	\item Reclutare i membri del progetto;
	\item Completare il \textit{Project Description Document};
	\item Stabilire le regole operative del team;
	\item Stabilire il processo per la gestione del cambiamento dello Scope;
	\item Gestire le comunicazioni del team;
	\item Finalizzare le schedule del progetto;
	\item Finalizzare i "work packages" (si tratta di componenti in cui è decomposto il lavoro da svolgere).
\end{itemize}
\subsubsection{Monitoring \& Controlling Process Group}
\begin{info}
	Si verifica lo stato di avanzamento del progetto rispetto al piano. Si risponde ad eventuali richieste di modifica e si risolvono le situazioni problematiche che emergono durante il progetto per garantire il progresso e il successo.
\end{info}
I processi consentono di:
\begin{itemize}
	\item Stabilire quali performance del progetto sono di interesse (KPI);
	\item Stabilire il sistema di reporting;
	\item Monitorare le performance del progetto e il rischio;
	\item Rendicontare lo stato del progetto;
	\item Elaborare le richiesta di modifica dello scope;
	\item Scopire e risolvere problemi.
\end{itemize}
\subsubsection{Closing Process Group}
\begin{info}
	Ci si assicura che siano stati soddisfatti tutti i requisiti del committente. Si provvede alla consegna dei deliverable.
\end{info}
I processi consentono di:
\begin{itemize}
	\item Pianificare ed eseguire l'installazione/consegna dei deliverables;
	\item Ottenere l'approvazione del committente per quanto riguarda il soddisfacimento del progetto;
	\item Scrivere il rapporto finale del progetto;
	\item Condurre una verifica post-implementazione del progetto.
\end{itemize}
\subsection{Definizione delle 9+1 Aree di Conoscenza (Knowledge Areas)}
Tutti questi processi si possono raggruppare in maniera diversa, invece che in fasi in cui entrare in azione si possono raggruppare per elementi da gestire durante il progetto, ovvero le "Knowledge Areas".
\subsubsection{Integration Management}
Riguarda i processi che consentono di effettuare il collegamento tra i diversi deliverables. Gli obiettivi devono sempre rimanere integrati tra loro per poter arrivare al goal di progetto. I processi coinvolti sono:
\begin{itemize}
	\item Sviluppo del Project Overview Statement;
	\item Sviluppo del Scope Statement;
	\item Sviluppo del Project Plan;
	\item Avvio dell’implementazione (esecuzione);
	\item Monitoraggio e controllo del lavoro svolto nel progetto;
	\item Controllo integrato dei cambiamenti;
	\item Chiusura del progetto (test di tutte le integrazioni);
	\item Altro...
\end{itemize}
\subsubsection{Scope Management}
Il cliente ci dovrebbe elencare i suoi bisogni, ma quello che fa è elencare i suoi desideri. Bisogna estrapolare tra tutti i desideri ciò che ci serve. Come fare? Identificando se nella richiesta c'è business value e quanto è. Non sempre il business value è facilmente misurabile.\newline
Per fare Scope Management è importante il concetto di \textit{Condition of Satisfaction}. C'è molta incertezza su quello che è il concetto reale, ognuno ha una visione diversa. In questo corso la visione che seguiamo è la seguente: la differenza tra una Condition of Satisfaction e un requisito è che la CoS va un po' oltre. Si tratta di un processo che tende a chiarire cosa vuole l'azienda e quindi verranno tradotti poi in requisiti. Nelle CoS ci sono anche i vincoli. Si tratta quindi delle condizioni che vanno ben oltre i requisiti non funzionali. Il processo consiste, a partire da una richiesta del cliente, nel capirla, rispondere all'esigenza con un "no" o con una proposta. Servirà poi concordare sulla qualità della risposta e capire se ci sono state incomprensioni, eventualmente ricominciando da capo.
\centeredImage{document/img/cos.PNG}{Condition of Satisfaction}{0.4}
\noindent Una volta determinati gli obiettivi e i requisiti si può utilizzare la \textbf{Requirements Breakdown Structure}, una struttura gerarchia che parte dal goal alla radice e scende via via dividendo il goal in Requirements, Function, Sub-Funcion e Feature. La struttura gerarchica ha il vantaggio di essere di molto semplice comprensione anche per il cliente o comunque una figura non tecnica.
\centeredImage{document/img/rbs.PNG}{Requirements Breakdown Structure}{0.5}
\subsubsection{Time Management}
Gestire il tempo è complicato. Immaginiamo una condizione ideale, ovvero quella rappresentata dalla linea tratteggiata, in cui lavoriamo ad un ritmo costante e lineare per produrre. Questo è valido solo assumendo di utilizzare tutto il tempo (giorno uomo di 8h) per produrre. Nella realtà molto tempo si perde per i più svariati motivi, quindi applichiamo un fattore di $0.75$. Esistono poi le unplanned interruptions (una chiamata imprevista etc.). La stima è che queste interruzioni portano via circa il $33\%$ del tempo.
\centeredImage{document/img/timemanag.PNG}{Perdita di tempo}{0.5}
Per stimare in modo corretto la durata delle attività ci sono vari modi:
\begin{itemize}
	\item Somiglianze con altre attività o utilizzo di dati storici;
	\item Pareri di esperti (che hanno lavorato spesso al problema). Può anche essere possibile chiedere pareri ad un team;
	\item Delphi technique;
	\item Three point technique;
	\item Wide band Delphi technique.
\end{itemize}
\subsubsection{Cost Management}
La gestione dei costi si può iniziare da quando si hanno più o meno indicazioni sui tempi necessari. La quantità di lavoro da fare è proporzionale al costo delle persone che servono a raggiungere l'obiettivo. Non si tratta dell'unico costo ma sicuramente del più oneroso, altri sono dovuti ai beni strumentali, ai servizi appaltati, alle spese di trasferta.\newline
Lo scopo è calcolare i costi in modo da poter generare un prezzo che possa anche far guadagnare l'azienda.
\subsubsection{Quality Management}
In un progetto ci sono due tipi di qualità:
\begin{itemize}
	\item \textbf{Qualità del prodotto}: Quanto il deliverable è adatto all'uso previsto.
	\item \textbf{Qualità del processo}: La bontà del processo di gestione del progetto.
\end{itemize}
Serve creare processi per poter garantire entrambe le qualità. Al Project Manager interessa di più la qualità del processo di cui è responsabile ma non può scordarsi di quella di prodotto.\newline
Il primo passo è pianificare la qualità, dopodiché, una volta ottenuto un piano di qualità dobbiamo essere sicuri di star eseguendo tutto nel modo giusto ed eventualmente effettuare operazioni di controllo per poterla ristabilire.
\centeredImage{document/img/qualitymanag.PNG}{Quality management}{0.5}
\paragraph{Quality Planning}
Determinare gli standard di qualità può essere complesso, il punto è capire su quale aspetto di qualità concentrarci. Fattori di input sono i fattori ambientali esterni ed interni, il Project Overview Statement e il Project management Plan.\newline
Il piano di qualità redatto deve documentare come le politiche verranno soddisfatte (il come vanno scritti i commenti, cosa è obbligatorio inserire, etc.), quali metriche utilizzare per la misurazione (coverage, test, etc.) ed un programma di miglioramento dei processi.
\paragraph{Quality Assurance}
Dobbiamo effettuari controlli, ma anche metterci nelle condizioni di poter seguire il piano di qualità. La Quality Assurance può prevedere:
\begin{itemize}
	\item Quality Audits (mi siedo col dev e faccio controlli sul codice);
	\item Process Analysis (analisi dei processi, la revisione serve per poterli migliorare;
	\item Project Quality Management Tools (dotarsi di strumenti in grado di poter valutare la qualità, ad esempio questionari).
\end{itemize}
\paragraph{Quality Control}
Prevede il monitoraggio delle performance del progetto per stabilire la conformità agli standard precedentemente adottati. Si definiscono le azioni da intraprendere se non vi è conformità con alcuni standard di qualità.
\subsubsection{Human Resources Management}
Abbiamo visto due tipi di organizzazione: Surgical Team e Scrum team. In ogni caso esistono dei profili chiave, in generale le figure presenti sono:
\begin{itemize}
	\item Developer co-manager;
	\item Client co-manager;
	\item Core Team;
	\item Task Leaders;
	\item Team Members.
\end{itemize}
L'attività consiste nel definire con accuratezza le conoscenze, le abilità e le competenze delle figure. Lavorando utilizzando un approccio Agile in cui ci sono meno figure specialistiche e le conoscenze sono più generali la gestione delle risorse umane diventa più semplice.
\paragraph{La teoria di Herzberg}
Dopo degli attenti studi, questo tizio ha concluso che ci sono dei fattori motivanti che muovono gli individui e li rendono più produttivi e più invogliati in generale:
\begin{itemize}
	\item Realizzazione;
	\item Riconoscimento;
	\item Progressione di carriera e crescita professionale;
	\item Responsabilità;
	\item Il lavoro stesso.
\end{itemize}
Inoltre ci sono dei "fattori di igiene", ovvero delle condizioni che non possono e non devono mancare:
\begin{itemize}
	\item Politiche Aziendali (per esempio non si dovrebbero utilizzare fondi aziendali invece che spese personali);
	\item Pratiche Amministrative (buona gestione per esempio dei salari);
	\item Condizioni di lavoro (un open space in cui c'è la temperatura corretta etc.);
	\item Supervisione tecnica (avere aiuto da figure in grado di valorizzare chi non conosce qualcosa, che non deve quindi imparare da autodidatta);
	\item Relazioni interpersonali (grazie alle buone relazioni tra le persone è possibile risolvere più facilmente i problemi);
	\item Sicurezza del lavoro;
	\item Salario.
\end{itemize}
\paragraph{Motivating Factors by J. Daniel Couger}
Un'opinione un po' diversa, da parte di J.D Couger che ha stilato una lista di fattori motivazionali ma ordinati dal più importante al meno importante:
\begin{itemize}
	\item Il lavoro stesso;
	\item Opportunità di realizzazione;
	\item Opportunità di progressione di carriera;
	\item Salario e benefit;
	\item Riconoscimento;
	\item Crescente responsabilità;
	\item Supervisione tecnica;
	\item Relazioni interpersonali;
	\item Sicurezza del lavoro;
	\item Condizioni di lavoro;
	\item Politiche aziendali.
\end{itemize}
\paragraph{Ruolo del Project Manager}
Le azioni del Project Manager influenzano pesantemente i fattori motivazionali dei membri del team. I fattori più influenti e che il Project Manager deve condividere sono:
\begin{itemize}
	\item Challenge (l'obiettivo, il cercare di capire quali sono i risultati per dare un senso a ciò che fanno);
	\item Recognition;
	\item Job Design:
	\begin{itemize}
		\item Skill Variety;
		\item Task Identity;
		\item Task Significance;
		\item Autonomy;
		\item Feedback.
	\end{itemize}
\end{itemize}
\subsubsection{Communications Management}
La comunicazione è un processo. L'informazione trasmessa da una persona all'altra deve essere trasmessa correttamente per essere compresa. L'informazione deve essere compresa e accettata dal destinatario. Un messaggio inviato è un messaggio ricevuto, spesso bisogna stare attenti perché se qualcosa cambia l'unico modo per fare in modo per emettere la nuova informazione è inviarla di nuovo.\newline
Per essere sicuri che il destinatario ha ricevuto e compreso un messaggio si può richiedere un feedback.
\centeredImage{document/img/comm.PNG}{Il processo di Communications Management}{0.5}
\paragraph{Tipi di comunicazione}
Esistono diversi tipi di comunicazioni:
\begin{itemize}
	\item One-to-one:
	\begin{itemize}
		\item Conversazioni (di persona, al telefono etc.)
		\item Riunioni e meeting
	\end{itemize}
	\item Elettronica:
	\begin{itemize}
		\item E-mails
		\item Websites
		\item Databases
	\end{itemize}
	\item Scritta:
	\begin{itemize}
		\item Memos
		\item Lettere
		\item Documenti
		\item Reports
	\end{itemize}
\end{itemize}
\paragraph{Interfacce di comunicazione}
\centeredImage{document/img/comminterface.PNG}{Interfacce di Communications Management}{0.5}
\begin{itemize}
	\item \textbf{Sponsor}: Può essere il cliente o chiunque vuole fortemenete il progetto, può essere una sola persona fisica o anche un gruppo di persone, un'azienda, etc.\newline
	Questa è la figura che ha capito la necessità del progetto ed eventualmente anche chi valuta se il risultato è stato raggiunto oppure no.
	\item \textbf{Team Managers}: I responsabili che gestiscono i vari team, anche fuori dal progetto.
	\item \textbf{Client}: Qui ci sono le persone che utilizzeranno il prodotto finale.
	\item \textbf{3$^{rd}$ Parties}: Spesso nei progetti utilizziamo dei fornitori esterni.
	\item \textbf{Project Team Members}: Il team che si occupa del progetto.
	\item \textbf{Public}: Utenti finali all'infuori dei clienti (es. chi utilizza il nostro sito di e-commerce).
	\item \textbf{Project Manager}: Il Project Manager deve avere una visione complessiva di tutte le comunicazioni, per questo sta in mezzo, ma non si deve occupare lui di tutte le comunicazioni perché questo causa ritardi (collo di bottiglia).
\end{itemize}
\subsubsection{Risk Management}
La gestione del rischio ha un suo ciclo di vita come ogni altro processo.
\begin{enumerate}
	\item \textbf{Identificazione dei rischi}: Risponde alla domanda \textit{"quali sono i rischi?"}\newline
	Si inizia a fare identificazione del rischio da subito, sin dallo scope del progetto. L'identificazione dei rischi però viene svolta anche durante l'esecuzione del progetto (effettuando riunioni etc.). I rischi da considerare sono:
	\begin{itemize}
		\item \textbf{Rischi tecnici}: vogliamo/dobbiamo utilizzare una tecnologia ma non sappiamo se porterà agli obiettivi sperati, o se è adatta per raggiungerli.
		\item \textbf{Rischi di Project Management}: dovuti ad una scarsa organizzazione, scarsa cura di dove vengono allocate le risorse, struttura di gestione inadeguata, indisciplina ed inesperienza.
		\item \textbf{Rischi di organizzazione}: problemi con la definizione di priorità, finanziamenti inadeguati, conflitti con altri progetti, gestione poco efficiente delle politiche aziendali etc.
		\item \textbf{Rischi esterni}: modifica dei requisiti legali e normativi, scioperi, problemi con i fornitori, appalti esterni, risorse necessarie che team esterni devono fornire etc.
	\end{itemize}
	\item \textbf{Valutazione dei rischi}: Risponde alle domande: "qual è la probabilità di perdite associate ai rischi trovati? Quanto costano le perdite? Quali sono le perdite nel caso peggiore?"\newline
	Chi fa Risk Management di solito cerca di calcolare la probabilità che qualcosa vada male (non sempre è possibile con stime accurate). Si calcola poi quanto vale il costo di un rischio e si fa fronte al rischio mettendo da parte dei soldi appositamente per quando servono. Se c'è una probabilità un rischio c'è sempre. La valutazione del rischio può essere di tipo:
	\begin{itemize}
		\item Statico: viene svolta solo all'inizio del progetto.
		\item Dinamico: viene svolta all'inizio del progetto e aggiornata in corso d'opera. Consente una gestione migliore del rischio ma è anche più costoso.
	\end{itemize}
	Esistono strutture a matrice che incrociano il tipo di rischio con gli elementi del triangolo di scope per specificare dove i rischi sono più alti. Il nome di questo schema è Risk Matrix Tool. Questa struttura non è sempre molto adatta perché alcuni rischi influiscono su più elementi. Strumenti visuali più utilizzati sono ad esempio la Risk Matrix che segue, dove su un asse c'è il rischi di perdita e nell'altro la probabilità che il rischio ci sia:
	\centeredImage{document/img/riskmatr.PNG}{Risk Matrix}{0.5}
	Esiste anche una struttura chiamata Quantitative Risk Assessment Worksheet che da un punteggio ad ogni incrocio tra attività e fattore di rischio (su una scala scelta) per valutare quali attività sono più rischiose o in generale l'impatto che il rischio può avere sul progetto:
	\centeredImage{document/img/worksheet.PNG}{Quantitative Risk Assessment Worksheet}{0.5}
	\item \textbf{Mitigazione dei rischi}: Risponde alle domande: "quali sono le alternative? Le alternative comportano rischi ulteriori? Come possono essere eliminate o ridotte le perdite?"\newline
	Per mitigare un rischio ci sono più modi:
	\begin{itemize}
		\item \textbf{Accettarlo}: significa non fare niente, non apportare modifiche. L'impatto è abbastanza basso da non essere significativo.
		\item \textbf{Evitare}: per esempio se utilizzare una nuova tecnologia comporta un rischio una possibile soluzione è non usare quella tecnologia.
		\item \textbf{Piano di contingenza}: si sceglie come ci si comporta solo quando e se il problema si presenta.
		\item \textbf{Mitigazione}: si sceglie come ci si comporta sin da subito, prima ancora che il problema si presenti.
		\item \textbf{Trasferimento}: si trasferisce l'impatto dell'evento avverso a terzi.
	\end{itemize}
	\item \textbf{Monitor e controllo dei rischi}: Un template utilizzabile è il Risk Log, ovvero una tabella in cui si annotano i rischi riscontrati, le azioni da intraprendere, i risultati, etc.\newline
	Lo scopo del Risk Log è anche favorire le attività future, in modo da valorizzare il miglioramento continuo.
	\centeredImage{document/img/risklog.PNG}{Risk Log}{0.5}
\end{enumerate}
\subsubsection{Procurement Management}
Si tratta della gestione dei fornitori. Anche questo è un processo importante:
\centeredImage{document/img/riskprocess.PNG}{Procurement Management - Ciclo di vita}{0.5}
\begin{itemize}
	\item \textbf{Vendor Sollecitation}: In questa fase si inquadrano i possibili vendor. Sviluppando l'RBS (o comunque l'insieme dei requisiti), potremmo aver bisogno di \textit{vendor} o soluzioni di terze parti. Identificare i fornitori non è semplice. Bisogna redigere un documento chiamato \textit{Request for Proposal (RFP)}, ovvero un documento da inviare ai fornitori per chiedere loro se sono interessati alla richiesta. Questo documento deve essere il più dettagliato possibile, facile da leggere, comprensibile. Bisogna anche scrivere in modo da voler invogliare i venditori a fornirci il loro servizio. Anche loro vogliono guadagnarci e di sicuro non si divertono a giocare a chi sta più basso con i prezzi. Di solito è preparato dal Procurement Office ed è supervisionato da un ufficio legale. Il RFP di solito è strutturato in un certo modo:
	\begin{itemize}
		\item Introduzione
		\item Business Profile
		\item Descrizione del problema/opportunità (opzionale)
		\item Project Overview Statement (POS) e/o (RBS) (opzionali, dipende a quale livello politico vogliamo mettere il fornitore, se vogliamo che sia più distaccato possiamo nascondergli i motivi per cui ci serve il suo servizio)
		\item Stime dei costi, dei tempi e del prezzo previsto, stimato in ordini di grandezza (per far capire più o meno le aspettative desiderate)
		\item Criteri utilizzati per valutare le offerte in fase di selezione (specificando cosa viene valutato il fornitore cercherà di valorizzare quegli aspetti)
		\item Descrizione della responsabilità del Vendor (il trasferimento delle penali va specificato)
		\item Dettagli amministrativi del contratto
		\item Istruzione per i vendors e contatti di riferimento o contatti per chiedere info aggiuntive
	\end{itemize}
	\textbf{Come si scelgono i vendors?}\newline
	Esiste un documento denominato \textbf{Request for Information (RFI)} che si può inviare ai vendors per avere chiarimenti ulteriori sul target dell'RFP. Un'altra opzione può essere la \textbf{pubblicità}, attirando i vendors attraverso offerte presenti su canali specifici. Questo di solito capita quando non si hanno contatti diretti con fornitori. I vendors che hanno dato buoni risultati in passato sono sempre un ottimo punto di riferimento. Partecipare a fiere o business meeting permette di conoscere altri fornitori disponibili sul mercato.
	\item \textbf{Vendor Evaluation}: Una volta ottenuta una lista dei vendors è necessario scegliere il più adatto, ma prima è necessario valutarli per "scremare" la scelta. Questo è possibile valutando i fornitori, richiedendo demo o prestazioni, valutando le risposte date a questionari etc.\newline
	Anche in questi casi esistono struttura utilizzabili per valutare i vendor, di solito questo lavoro è ad opera di alcuni consultant, ovvero figure che hanno il ruolo di dare una votazione ai fornitori. La scelta del fornitore si basa anche in questo caso su indici:
	\centeredImage{document/img/forcedranking.PNG}{Forced Ranking}{0.5}
	In questo caso serve anche avere delle motivazioni sui voti per compiere scelte e non è sufficiente affidarsi solo ai numeri.\newline
	Si può anche valutare in base a quale prodotto è meglio degli altri. Nella valutazione \textit{Paired Comparison} si confrontano i prodotti a due a due.
	\centeredImage{document/img/pairedcomparison.PNG}{Paired Comparison}{0.5}

	\item \textbf{Vendor Selection}: L'esito della selezione può avere tre tipi di risultato:
	\begin{itemize}
		\item \textbf{No Award}: nessun vendor soddisfa i requisiti minimi. Questo significa che bisogna ripartire da capo o rinunciare al progetto. In alcuni casi si può anche rinunciare solo al fornitore e sviluppare la soluzione direttamente in azienda se si scopre essere particolarmente vantaggiosa.
		\item \textbf{Single Award}: è stato selezionato un unico fornitore.
		\item \textbf{Multiple Award}: sono stati selezionati più fornitori che si dividono il lavoro per soddisfare i requisiti.
	\end{itemize}
	\item \textbf{Vendor Contracting}: Si negoziano le condizioni del contratto coi vendors.
	\item \textbf{Vendor Management}: Si definiscono le fasi iniziali della collaborazione con il fornitore. Si monitorano i progressi e le performance. Si eseguono i test di accettazione.
\end{itemize}
\subsubsection{Stakeholder Management}
Gli agilisti si sono accorti che gli stakeholders sono molto più importanti di come si è sempre creduto.
PMBOK inizialmente prevedeva solo due processi nell'area di conoscenza Communications Management, ora sono diventati 4:
\begin{itemize}
	\item \textbf{Identificazione degli Stakeholders}: Capire chi sono e quali sono i loro interessi, le interrelazioni, cosa cercano nel progetto.
	\item \textbf{Pianificazione la gestione dello Stakeholders}: Si pianificano le strategie per andare incontro agli interessi degli Stakeholders.
	\item \textbf{Gestione degli Stakeholders}: Si comunica e lavora con gli stakeholders per arrivare ad un risultato.
	\item \textbf{Controllo degli Stakeholders}: Correggiamo gli aspetti che non funzionano durante il progetto.
\end{itemize}
\paragraph{Stakeholders Analysis}
L'idea è raccogliere informazioni per capire chi sono gli interlocutori. Si identificano gli interessi, le aspettative, l'influenza etc. dopodiché bisogna ordinare gli stakeholders per "importanza". Il nostro desiderio è convincere i contrari e far leva sull'entusiasmo dei positivi.
\centeredImage{document/img/stakeholders.PNG}{Visione degli Stakeholders di PMBOK}{0.5}
\noindent Le persone che partecipano attivamente al progetto, il \textbf{Project Team} rappresentano lo Stakeholder di default. Lo \textbf{sponsor} è chi investe nel progetto e ha un'esigenza. Gli \textbf{Operation Manager} sono le persone che gestiscono il funzionamento dell'azienda. I \textbf{Functional Manager} sono i responsabili di amministrazione, di logistica... I \textbf{Partner} sono le aziende a noi alleate. Il \textbf{Manager del Portfolio} è il manager di tutti i progetti in generale.\newline
Un metodo per gestire tutti gli Stakeholders è usare la Stakeholders Matrix. Sull'asse della $x$ c'è l'interesse degli Stakeholders per il progetto mentre nelle $y$ c'è l'influenza sul progetto degli Stakeholders. Ovviamente gli Stakeholders nel quadrante rosso (Key Player) sono molto importanti. Il quadrante giallo rappresenta stakeholders con molta influenza ma poco interesse, magari perché non c'è per loro un valore diretto o un effettivo vantaggio. Queste figure non vanno ignorate, in genere si cerca di spostare queste figure verso destra, in modo che anche loro possano capire il valore del progetto e mettere anche più budget se necessario.\newline
Il giallo chiaro rappresenta gli Stakeholders che non hanno potere ma sono fortemente interessati. Il nostro scopo è cercare di farli sentire partecipi e ascoltare i loro bisogni. La raccomandazione di come vanno trattati gli Stakeholders verdi è cercare di portarli più a destra per lo stesso motivo di quelli gialli scuri.
\centeredImage{document/img/stakeholdermatrix.PNG}{Stakeholders Matrix}{0.6}
\subsection{Mappatura delle Aree di Conoscenza nei Gruppi di Processi}
Esiste un matching di classificazione anche sulle aree di conoscenza. L'aspetto negativo resta il voler classificare a tutti i costi. Infatti alcuni gruppi potrebbero essere considerati parte di più knowledge area.
\centeredImage{document/img/maping.PNG}{Mapping delle aree di conoscenza nei gruppi di processi}{0.7}
