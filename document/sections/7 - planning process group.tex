\section{Planning Process Group}
\subsection{L’importanza della pianificazione}
La curva della sofferenza serve per dare una raccomandazione. La pianificazione non è facile da fare, richiede tempo e pazienza. L'inizio del progetto è molto invitante, ma ci sono diversi problemi a partire senza pianificare. Senza pensare alle attività non avremo idea di come verranno collocate nel tempo e non sapremo quanto dureranno. Avere un piano ci consente anche di sapere quali e quante risorse abbiamo a disposizione.
\centeredImage{document/img/paincurve.png}{Pain Curve secondo Wysocki}{0.5}
\noindent Con un piano, se dopo una settimana siamo in ritardo mi farò delle domande su cosa non sta funzionano. Senza un piano non so neanche di essere in ritardo, e la domanda viene posticipata...
Le scadenze sono il miglior metodo di monitoraggio.

\noindent La pianificazione:
\begin{itemize}
	\item \textbf{riduce l'incertezza}: stabiliamo a priori attività, chi le deve fare, si capiscono i tempi, quando finisce il progetto, come dovrebbe andare il flusso di cassa etc.
	\item \textbf{aumenta la comprensione}: Pensando a come implementare la soluzione ci si rende conto che molti dettagli che sembravano ovvi in realtà non lo sono.
	\item \textbf{migliora l'efficienza}: sapendo chi deve fare cosa si può meglio organizzare tutto quanto in modo che non ci siano "momenti morti".
\end{itemize}

\subsection{Tools, templates e processi per pianificare un progetto}
Ci sono un sacco di soluzioni software anche in cloud per poter supportare il processo di gestione di un progetto:
\begin{itemize}
	\item \textbf{Jira} per i progetti AGILE
	\item \textbf{Slack} per la comunicazione
	\item \textbf{Trello} per la gestione di task e card
\end{itemize}
Ognuno ha le sue caratteristiche ed i suoi obiettivi, oltre alle sue preferenze.
\subsubsection{Tool fisici}
Le lavagne appese ad una parete con gli sticky notes appesi in giro sono ancora oggi strumenti molto utilizzati, dato che i software non permettono ancora lo stesso grado di flessibilità. I colori dei pennarelli e delle lavagne sono molto utili perché simboleggiano diverse informazioni. Gli sticky note possono essere usati nel seguente modo. Ogni sticky node può essere relativo ad un'attività e può avere un ID, un nome ed una descrizione. Devono essere utilizzati per fornire informazioni rapide e non possono contenere tanti dati. Possono anche contenere il responsabilità di un'attività, le risorse richieste. Servono anche la quantità di lavoro e la durata dell'attività, perché mi danno un quadro generale delle stato delle risorse. La quantità mi da informazioni sui giorni uomo che mi servono e quindi indirettamente il costo. La durata mi serve per mettere a calendario l'attività. Vengono annotati anche Ealiest Start, Earliest Finish, Latest Start, Latest Finish, che sono valori calcolati.

\noindent Le lavagne e i pennarelli vengono utilizzati per rappresentare dipendenze e ordini temporali tra le attività, e ci permettono di identificare il percorso critico.
\centeredImage{document/img/marking.png}{Utilizzo dei pennarelli}{0.5}
Esistono tanti tipi di lavagne, si pososno preparare in base a template etc.
Le informazioni da tenere a portata di mano attaccati sulla lavagna ci sono il POS, la WBS, il diagramma delle dipendenze, project schedule, resource schedule, issue log (problemi che emergono durante il progetto) etc.
\subsection{Pianificare e condurre la Joint Project Planning Session (JPPS)}
\subsection{Scrivere un Project Description Statement (PDS)}
\subsection{Costruire la Work Breakdown Structure (WBS)}
\subsection{Stimare la durata delle attività (estimating task duration)}
\subsection{Stimare il fabbisogno di risorse (estimating resource requirements)}
\subsection{Stimare i costi (estimating cost)}
\subsection{Costruire il project network diagram}
\subsection{Scrivere una proposta di progetto efficace}
\subsection{Ottenere l’approvazione per passare alla fase di esecuzione (launch/execution)}

\newpage