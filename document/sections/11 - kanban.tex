\section{Kanban e altro ancora...}
\subsection{Kanban}
Kanban viene dal concetto di "lean manufacturing". Il mondo Toyota ha identificato un modo per produrre in modo migliore. In generale se dobbiamo classificare la metodologia è un modo flessibile produrre just in time. Nell’ambito della produzione di beni, l’obiettivo del lean manifacturing è limitare le scorte, aumentare la flessibilità del sistema produttivo e dare una risposta rapida alle richieste del mercato. La conseguenza è che si provvedere a “procurarsi” le risorse necessarie (componenti, materie prime, etc.) solo in presenza di un ordine e solo all’ultimo momento utile per andare in produzione.
Il vantaggio è non fare magazzino, risparmiando anche sulla metratura quadra necessaria per depositare fisicamente le risorse. Gli stessi principi li possiamo applicare al software.
Analogamente all'AGILE si tratta di un approccio in cui i rischi non sono pochi e serve mantenere un certo livello di disciplina.

\subsubsection{Le sei regole di Toyota}
\begin{itemize}
	\item \textbf{Non passare mai prodotti che hanno difetti}: I test sono fondamentali, se c'è un difetto non si passa in produzione. Un'automobile che non funziona non si vende;
	\item \textbf{Prendi solo quel che serve}: In produzione viene messo solo ciò che serve;
	\item \textbf{Produci esattamente la quantità necessaria}: Vero anche per il software. Si rispettano i requisiti. Non una riga di codice in più;
	\item \textbf{Mantenere il passo}:
	\item \textbf{Fare il fine-tuning della produzione}: la retrospettiva è fondamentale per migliorare il processo produttivo futuro;
	\item \textbf{Stabilire e razionalizzare il processo}.
\end{itemize}
\subsection{Kanban \& Kaizen}
Kaizen è il processo di miglioramento continuo, implica l’analisi critica e la revisione
costante dei processi aziendali. Negli approcci agili è importante effettuare un’analisi retrospettiva ad
ogni iterazione per migliorare i processi di gestione e produzione.
\subsection{Project Management: l’approccio Kanban}
Kanban è un approccio di project management utilizzato in ambito agile e DevOps.
Consente di supportare il “miglioramento continuo” e aiuta i team che sviluppano “prodotti” e “servizi” a selezionare i giusti carichi di lavoro e a completare i task rispettando le priorità.
Il termine “kanban” spesso identifica anche la “kanban board”, che è il suo strumento di base, utilizzato anche in altri approcci agile. La kanban board consente di controllare il numero dei task in esecuzione
(limit work-in-progress) massimizzando l’efficienza del “flusso” di lavoro(flow).\newline
Il Project Management Life Cycle (PMLC) è “iterativo”, ma a differenza di altri approcci (e.g., Scrum) non prevede la definizione formale di uno schema “sincrono” di iterazioni.
\subsection{Kanban Board}
La Kanban board è composta da due elementi principali:
\begin{itemize}
	\item \textbf{Visual Signals}: Si tratta delle cosiddette card. Possono essere dei post-it attaccati ad una lavagna divisa per colonne. All'interno delle card viene riportato del lavoro da svolgere. Possiamo fare un parallelismo con un team Agile che inserirebbe delle user story nelle card. Non c'è uno standard su come le card devono essere rappresentate.
	\item \textbf{Colonne}: Ogni colonna rappresenta uno stato del workflow in cui può trovarsi una card. Esempi classici sono "To Do", "In Progress", "Complete", ma si può complicare a piacere.
\end{itemize}

Aspetti fondamentali della Kanban Board:
\begin{itemize}
	\item \textbf{Limiti su Work in progess (WIP)}: Non bisognerebbe avere troppe card in produzione. Nella colonna "In Progress" non bisognerebbe mai superare un certo limite, che si sceglie in base alle risorse disponibili.
	\item \textbf{Commitment Point}: Il punto in cui si inizia a lavorare su un idea del customer o di un membro del team, che quindi inizierà a concretizzarsi.
	\item \textbf{Delivery Point}: Il punto in cui si consegna un risultato atteso. Coincide praticamente con l'ultimo passo del workflow della Board.
\end{itemize}
\begin{info}[Lead Time:]
	Lo scopo del team è passare dal Commitment Point al Delivery Point il più velocemente possibile. Il tempo che intercorre tra questi due momenti si chiama \textit{Lead Time}.
\end{info}

\subsubsection{Esempi di Template}
In quetso esempio le colonne sono "Requested", "In Progress" in cui il WIP vale 3 e "Done". La Swimlane è l'equivalente dei tag per le card (quindi per categorizzarle e dividerle meglio).
\centeredImage{document/img/kanban1.png}{Esempio di Template Kanban \#1}{0.4}
\noindent Idea leggermente più complessa consiste nell'avere un Backlog, delle colonne "In Progress" "In Review" e "In Test" che hanno un WIP seguita da un "Done". La colonna "Blocked" vuol dire che non è più possibile toccare quella card ed effettuare modifiche.
\centeredImage{document/img/kanban2.png}{Esempio di Template Kanban \#2}{0.4}
\noindent In questo caso le "Stories" rappresentano idea che possono passare a "To Do" solo se approvate.
\centeredImage{document/img/kanban3.png}{Esempio di Template Kanban \#3}{0.4}
\noindent È possibile complicare le board a piacere, categorizzando anche le colonne.
\centeredImage{document/img/kanban4.png}{Esempio di Template Kanban \#4}{0.7}
\noindent Il lead time in questo caso è il tempo entro cui la card deve arrivare a "Done" dal momento in cui viene creata e piazzata in "Ready to Start".
\centeredImage{document/img/kanban5.png}{Esempio di Template Kanban (Lead Time)}{0.7}
\noindent Il "Cycle Time" comincia a passare dal momento in cui si inizia a produrre. Sostanzialmente è una versione più specifica del "Lead Time". Diminuire il Cycle Time significa essere più produttivi.
\centeredImage{document/img/kanban6.png}{Esempio di Template Kanban (Cycle Time)}{0.7}

\subsection{Kanban vs Scrum}
\begin{itemize}
	\item \textbf{Kanban} consiste in un approccio basato sulla visualizzazione del lavoro da svolgere, limitando il lavoro in corso e massimizzando l'efficienza del flusso. I team che usano Kanban focalizzano la loro attenzione sul lead time e cycle time, cercando di migliorare continuamente il flusso di lavoro.
	\item \textbf{Scrum} impiega iterazioni di durata prefissata (sprint). Il loro obiettivo è creare cicli di apprendimento per raccogliere e integrare rapidamente i feedback dei clienti. I team di Scrum adottano ruoli specifici, creano artefatti (backlog, etc.) e tengono “cerimonie” regolari per procedere nello sviluppo e poter migliorare il risultato finale. Il problema di Scrum è che se viene calibrato male il lavoro è necessario aspettare la fine di uno sprint per ricalibrare il lavoro, problema che in Kanban è assente in quanto più flessibile sotto questo punto di vista.
\end{itemize}
\begin{info}
	I due approcci si focalizzano quindi su due aspetti diversi:
	\begin{itemize}
		\item \textbf{Kanban} è focalizzato sullo scope;
		\item \textbf{Scrum} è focalizzato sul tempo.
	\end{itemize}
\end{info}
\centeredImage{document/img/kanbanvsscrum.png}{Tabella delle differenze: Kanban vs Scrum}{0.5}
\subsection{L’uso dell’approccio Kanban}
L’approccio Kanban è adatto ad essere impiegato in progetti in cui c’è l’esigenza di iterare. In particolare, Kanban è adatto a quei progetti in cui si è focalizzati sullo scope, che non è chiaramente definito, e la gestione di cicli di durata costante è non praticabile. Due ambiti in cui l’uso di Kanban potrebbe essere adatto sono DevOps e progetti R\&D (Ricerca e Sviluppo), ma non limitato solo a questi.
Per quanto riguarda i progetti R\&D sono chiari i vantaggi:
\begin{itemize}
	\item Lo scope instabile può implicare la revisione continua degli item in lavorazione;
	\item Le revisioni non sono costanti nel tempo, ma irregolari e dipendono dai risultati ottenuti dagli altri item in lavorazione;
	\item Necessità di ridurre lead time e/o cycle time.
\end{itemize}

\subsection{Gestione delle risorse in Kanban}
Se gestiamo un progetto con un approccio Kanban, proviamo le domande a cui rispondere sono le seguenti:
\begin{itemize}
	\item \textbf{Come e quando si stabiliscono i tempi?};
	\item \textbf{Come e quando si stabiliscono le risorse?}: Si possono utilizzare diversi approcci, un esempio è il "Just in Time". Man mano che si libera una risorsa si può scegliere una nuova card su cui allocarla. Requisito necessario è avere un team sufficientemente versatile;
	\item \textbf{Come e quando si stabiliscono i costi?}: Un tema molto delicato per questo approccio, in quanto le card possono arrivare di continuo e non si hanno indicazioni chiare sullo scope che è molto instabile. Di solito in questi casi si paga in funzione del tempo lavorato.
	\item \textbf{Come può essere definito un contratto adatto a Kanban?}: Spesso consuntivo.
	\item \textbf{Quali altri aspetti dobbiamo considerare?}: dipende fortemente dalla specifica implementazione di Kanban.
\end{itemize}
